\begin{center}
\begin{tabular}{|l|} 
\hline
\\
Paritaria \textbf{se representa con} estrP, donde estrP es: \\
\tupla{\\
\hspace*{4em}\param{}{gremio}{puntero(gremio)},\\
\hspace*{4em}\param{}{piso}{nat},\\
\hspace*{4em}\param{}{tope}{nat} \\\hspace*{2em} } \\
\\
\hline
\end{tabular}
\end{center}

\subsubsection{Invariante de Representaci\'on}
\paragraph{El Invariante Informalmente}
\begin{enumerate}
\item El piso debe ser menor o igual que el techo.
\end{enumerate}

\paragraph{El Invariante Formalmente}
\begin{Rep}{estrP}{e}
\repfunc{e.piso \leq e.tope}{}
\end{Rep}

\subsubsection{Funci\'on de Abstracci\'on}
\begin{ABSEXPLICITO}{e}{estrP}{nuevaParitaria(*e.gremio, e.piso, e.tope)}{paritaria}
{}
\end{ABSEXPLICITO}