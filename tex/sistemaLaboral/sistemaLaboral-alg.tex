\algoritmoI{NuevoSistemaLaboral}{}{estrS}
{
	\state res.gremios \asig Vacía()					\inlineC{0}
	\state res.gruposDeAliados \asig Vacía()			\inlineC{0}
}
{0}
{Lorem ipsum dolor sit amet, consectetur adipiscing elit. Sed orci arcu, cursus sed consequat et, facilisis ac nunc. Donec posuere bibendum purus nec luctus. Etiam est erat, dictum eget vehicula eget, consectetur a ipsum. Sed non tortor magna. Lorem ipsum dolor sit amet, consectetur adipiscing elit. Mauris ullamcorper lacus ut ante lacinia mattis. Etiam dictum convallis neque, sit amet molestie nisi porta non.}

\algoritmoI{AgregarGremio}{\param{in/out}{sl}{estrS}, \param{in/out}{gr}{gremio}}{}
{
	\state \var{idGremio}{id} \asig ObtenerProximoIdGremio(sl)			\inlineC{0}
	\state
	\state GuardarId(gr,id)												\inlineC{0}
	\state sl.gremios[id] \asig \&gr									\inlineC{0}
	\state sl.gruposDeAliados[id] \asig ObtenerProximoIdGrupoDeAliados(sl)			\inlineC{0}
}
{0}
{Lorem ipsum dolor sit amet, consectetur adipiscing elit. Sed orci arcu, cursus sed consequat et, facilisis ac nunc. Donec posuere bibendum purus nec luctus. Etiam est erat, dictum eget vehicula eget, consectetur a ipsum. Sed non tortor magna. Lorem ipsum dolor sit amet, consectetur adipiscing elit. Mauris ullamcorper lacus ut ante lacinia mattis. Etiam dictum convallis neque, sit amet molestie nisi porta non.}

\algoritmoI{AliarGremios}{\param{in/out}{sl}{estrS}, \param{in}{gr1}{gremio}, \param{in}{gr2}{gremio}}{}
{
	\state \var{idGrupo}{idGrupoG1} \asig ObtenerIdGrupoDeAliados(sl, \&gr1)			\inlineC{0}
	\state \var{idGrupo}{idGrupoG2} \asig ObtenerIdGrupoDeAliados(sl, \&gr2)			\inlineC{0}
	\state 
	\state UnificarGruposDeAliados(sl, idGrupoG1, idGrupoG2)							\inlineC{0}
}
{0}
{Lorem ipsum dolor sit amet, consectetur adipiscing elit. Sed orci arcu, cursus sed consequat et, facilisis ac nunc. Donec posuere bibendum purus nec luctus. Etiam est erat, dictum eget vehicula eget, consectetur a ipsum. Sed non tortor magna. Lorem ipsum dolor sit amet, consectetur adipiscing elit. Mauris ullamcorper lacus ut ante lacinia mattis. Etiam dictum convallis neque, sit amet molestie nisi porta non.}

\algoritmoI{ObtenerGremios}{\param{in}{sl}{estrS}}{conj(puntero(gremio))}
{
	\state res \asig Vacío()								\inlineC{0}

	\state
	\state \var{nat}{indice} \asig 0						\inlineC{0}
	\while{indice \menor Longitud(sl.gremios)}				\inlineC{0}
		\state

		\state Agregar(res, sl.gremios[indice])			\inlineC{0}

		\state
		\state indice \asig indice + 1						\inlineC{0}
	\endwhile
}
{0}
{Lorem ipsum dolor sit amet, consectetur adipiscing elit. Sed orci arcu, cursus sed consequat et, facilisis ac nunc. Donec posuere bibendum purus nec luctus. Etiam est erat, dictum eget vehicula eget, consectetur a ipsum. Sed non tortor magna. Lorem ipsum dolor sit amet, consectetur adipiscing elit. Mauris ullamcorper lacus ut ante lacinia mattis. Etiam dictum convallis neque, sit amet molestie nisi porta non.}

\algoritmoI{Obtener\#GruposDeAliados}{\param{in}{sl}{estrS}}{nat}
{
	\state res \asig MaximoIdGrupoDeAliados(sl)			\inlineC{0}
}
{0}
{Lorem ipsum dolor sit amet, consectetur adipiscing elit. Sed orci arcu, cursus sed consequat et, facilisis ac nunc. Donec posuere bibendum purus nec luctus. Etiam est erat, dictum eget vehicula eget, consectetur a ipsum. Sed non tortor magna. Lorem ipsum dolor sit amet, consectetur adipiscing elit. Mauris ullamcorper lacus ut ante lacinia mattis. Etiam dictum convallis neque, sit amet molestie nisi porta non.}

\algoritmoI{ObtenerIdGrupoDeAliados}{\param{in}{sl}{estrS}, \param{in}{gr}{puntero(gremio)}}{idGrupo}
{
	\state res \asig sl.gruposDeAliados[ObtenerId(*gr)]			\inlineC{0}
}
{0}
{Lorem ipsum dolor sit amet, consectetur adipiscing elit. Sed orci arcu, cursus sed consequat et, facilisis ac nunc. Donec posuere bibendum purus nec luctus. Etiam est erat, dictum eget vehicula eget, consectetur a ipsum. Sed non tortor magna. Lorem ipsum dolor sit amet, consectetur adipiscing elit. Mauris ullamcorper lacus ut ante lacinia mattis. Etiam dictum convallis neque, sit amet molestie nisi porta non.}

\algoritmoI{GuardarIdGrupoDeAliados}{\param{in/out}{sl}{estrS}, \param{in}{gr}{puntero(gremio)}, \param{in}{id}{idGrupo}}{}
{
	\state sl.gruposDeAliados[ObtenerId(*gr)] \asig id			\inlineC{0}
}
{0}
{Lorem ipsum dolor sit amet, consectetur adipiscing elit. Sed orci arcu, cursus sed consequat et, facilisis ac nunc. Donec posuere bibendum purus nec luctus. Etiam est erat, dictum eget vehicula eget, consectetur a ipsum. Sed non tortor magna. Lorem ipsum dolor sit amet, consectetur adipiscing elit. Mauris ullamcorper lacus ut ante lacinia mattis. Etiam dictum convallis neque, sit amet molestie nisi porta non.}

\algoritmoI{ObtenerProximoIdGremio}{\param{in}{sl}{estrS}}{idGremio}
{
	\state res \asig Longitud(sl.gremios)			\inlineC{0}
}
{0}
{Lorem ipsum dolor sit amet, consectetur adipiscing elit. Sed orci arcu, cursus sed consequat et, facilisis ac nunc. Donec posuere bibendum purus nec luctus. Etiam est erat, dictum eget vehicula eget, consectetur a ipsum. Sed non tortor magna. Lorem ipsum dolor sit amet, consectetur adipiscing elit. Mauris ullamcorper lacus ut ante lacinia mattis. Etiam dictum convallis neque, sit amet molestie nisi porta non.}

\algoritmoI{ObtenerProximoIdGrupoDeAliados}{\param{in}{sl}{estrS}}{idGrupo}
{
	\If{Longitud(sl.gruposDeAliados) \igual 0}					\inlineC{0}
		\state res \asig 0										\inlineC{0}
	\Else
		\state res \asig MaximoIdGrupoDeAliados(sl) + 1			\inlineC{0}
	\endif
}
{0}
{Lorem ipsum dolor sit amet, consectetur adipiscing elit. Sed orci arcu, cursus sed consequat et, facilisis ac nunc. Donec posuere bibendum purus nec luctus. Etiam est erat, dictum eget vehicula eget, consectetur a ipsum. Sed non tortor magna. Lorem ipsum dolor sit amet, consectetur adipiscing elit. Mauris ullamcorper lacus ut ante lacinia mattis. Etiam dictum convallis neque, sit amet molestie nisi porta non.}

\algoritmoI{UnificarGruposDeAliados}
{\param{in/out}{sl}{estrS}, \param{in}{id1}{idGrupo}, \param{in}{id2}{idGrupo}}{}
{
	\state \var{idGrupo}{idMenor} \asig Min(id1, id2)						\inlineC{0}
	\state \var{idGrupo}{idMayor} \asig Max(id1, id2)						\inlineC{0}
	\state

	\state \var{nat}{indice} \asig 0										\inlineC{0}

	\while{indice \menor Longitud(sl.gremios)}								\inlineC{0}
		\state

		\state \var{puntero(gremio)}{gr} \asig sl.gremios[indice]			\inlineC{0}
		\state \var{idGrupo}{idG} \asig ObtenerIdGrupoDeAliados(sl, gr)		\inlineC{0}
		\state

		\If{idMayor \igual idG}												\inlineC{0}
			\state GuardarIdGrupoDeAliados(sl, gr, idMenor)					\inlineC{0}
		\Else \If{idMayor \menor idG}										\inlineC{0}
				\state GuardarIdGrupoDeAliados(sl, gr, idG - 1)				\inlineC{0}
			\endif
		\endif
		\state

		\state indice \asig indice + 1			\inlineC{0}
	\endwhile
}
{0}
{Lorem ipsum dolor sit amet, consectetur adipiscing elit. Sed orci arcu, cursus sed consequat et, facilisis ac nunc. Donec posuere bibendum purus nec luctus. Etiam est erat, dictum eget vehicula eget, consectetur a ipsum. Sed non tortor magna. Lorem ipsum dolor sit amet, consectetur adipiscing elit. Mauris ullamcorper lacus ut ante lacinia mattis. Etiam dictum convallis neque, sit amet molestie nisi porta non.}

\algoritmoI{MaximoIdGrupoDeAliados}{\param{in}{sl}{estrS}}{nat}
{
	\state \var{nat}{maxId} \asig 0									\inlineC{0}

	\state
	\state \var{nat}{indice} \asig 0								\inlineC{0}
	\while{indice \menor Longitud(sl.gruposDeAliados)}				\inlineC{0}
		\state
		
		\state maxId \asig Max(maxId, sl.gruposDeAliados[indice])	\inlineC{0}

		\state
		\state indice \asig indice + 1								\inlineC{0}
	\endwhile
	\state

	\state res \asig maxId											\inlineC{0}
}
{0}
{Lorem ipsum dolor sit amet, consectetur adipiscing elit. Sed orci arcu, cursus sed consequat et, facilisis ac nunc. Donec posuere bibendum purus nec luctus. Etiam est erat, dictum eget vehicula eget, consectetur a ipsum. Sed non tortor magna. Lorem ipsum dolor sit amet, consectetur adipiscing elit. Mauris ullamcorper lacus ut ante lacinia mattis. Etiam dictum convallis neque, sit amet molestie nisi porta non.}

\algoritmoI{Min}{\param{in}{id1}{idGrupo}, \param{in}{id2}{idGrupo}}{idGrupo}
{
	\If{id1 \menor id2}					\inlineC{0}
		\state res \asig id1			\inlineC{0}
	\Else
		\state res \asig id2			\inlineC{0}
	\endif
}
{0}
{Lorem ipsum dolor sit amet, consectetur adipiscing elit. Sed orci arcu, cursus sed consequat et, facilisis ac nunc. Donec posuere bibendum purus nec luctus. Etiam est erat, dictum eget vehicula eget, consectetur a ipsum. Sed non tortor magna. Lorem ipsum dolor sit amet, consectetur adipiscing elit. Mauris ullamcorper lacus ut ante lacinia mattis. Etiam dictum convallis neque, sit amet molestie nisi porta non.}

\algoritmoI{Max}{\param{in}{id1}{idGrupo}, \param{in}{id2}{idGrupo}}{idGrupo}
{
	\If{id1 \menor id2}					\inlineC{0}
		\state res \asig id2			\inlineC{0}
	\Else
		\state res \asig id1			\inlineC{0}
	\endif
}
{0}
{Lorem ipsum dolor sit amet, consectetur adipiscing elit. Sed orci arcu, cursus sed consequat et, facilisis ac nunc. Donec posuere bibendum purus nec luctus. Etiam est erat, dictum eget vehicula eget, consectetur a ipsum. Sed non tortor magna. Lorem ipsum dolor sit amet, consectetur adipiscing elit. Mauris ullamcorper lacus ut ante lacinia mattis. Etiam dictum convallis neque, sit amet molestie nisi porta non.}