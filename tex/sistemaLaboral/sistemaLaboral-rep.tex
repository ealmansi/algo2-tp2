\begin{center}
\begin{tabular}{|l|} 
\hline
\\
SistemaLaboral \textbf{se representa con} estrS, donde estrS es: \\
\tupla{\\
\hspace*{4em}\param{}{gremios}{vector(puntero(gremio))},\\
\hspace*{4em}\param{}{gruposDeAliados}{vector(idGrupo)} \\\hspace*{2em} } \\
\\
\hline
\end{tabular}
\end{center}

\subsubsection{Invariante de Representaci\'on}
\paragraph{El Invariante Informalmente}

\begin{enumerate}
\item Los gremios no comparten empresas.
\item Los vectores \emph{gremios} y \emph{gruposDeAliados} tienen igual longitud (una entrada por cada gremio agregado).
\item Los valores del vector \emph{gruposDeAliados} son consecutivos.

\end{enumerate}

\paragraph{El Invariante Formalmente}
% \begin{Rep}{estrAC}{ac}
% \repfunc{(\forall\param{}{c}{Categoria}) (def?(c,e.familia)) \Leftrightarrow (*obtener(c,e.familia)).categoria = c}{\yluego}
% \repfunc{(\forall\param{}{c_{1}}{Categoria}) (def?(c_{1},e.familia)) \Leftrightarrow ((c_{1} == e.raiz) \vee \\
% ((\exists\param{}{c_{2}}{Categoria})(def?(c_{2},e.familia)) \yluego c_{1} \in (*obtener(c_{2},e.familia)).hijos))}{\yluego}
% \repfunc{(\forall\param{}{c}{Categoria})(def?(c,e.familia) \Leftrightarrow\\ (((\exists\param{}{d}{datosCat})esta?(d,e.categorias)\land d.categoria==c)\yluego d==obtener(c,e.familia)))}{\yluego}
% \repfunc{(\forall\param{}{c_{1},c_{2}}{Categoria}) (def?(c_{1},e.familia)) \land (def?(c_{2},e.familia)) \Rightarrow_{L} \\ c_{2} \in *((obtener(c_{1},e.familia)) ).hijos \Leftrightarrow \\ (*(*(obtener(c_{2},e.familia))).padre).categoria = c_{1}}{\yluego}
% \repfunc{e.cantidad  = longitud(e.categorias)}{\yluego}
% \repfunc{(\forall\param{}{c}{categoria}) (def?(c,e.familia)) \land c = e.raiz \Rightarrow_{L} \\ (*(obtener(c,e.familia))).altura = 1 \land (*(obtener(c,e.familia))).id = 1}{\yluego}
% \repfunc{(\forall\param{}{c}{Categoria}) (def?(c,e.familia)) \Rightarrow_{L} (*obtener(c,e.familia)).altura \leq e.alturaMax}{\yluego}
% \repfunc{(\exists\param{}{c}{Categoria}) (def?(c,e.familia)) \yluego *((obtener(c,e.familia)) ).altura = e.alturaMax}{\yluego}
% \repfunc{(\forall\param{}{c_{1},c_{2}}{Categoria}) (def?(c_{1},e.familia)) \wedge (def?(c_{2},e.familia)) \yluego \\
% ((\exists\param{}{d}{datosCat})d\in (*(obtener(c_{1},e.familia))).hijos \land d.categoria == c_{2}) \Rightarrow_{L} \\
% (*(obtener(c_{2},e.familia))).altura = 1 + (*(obtener(c_{1},e.familia))).altura}{\yluego}
% \repfunc{(\forall\param{}{c}{Categoria}) (def?(c,e.familia)) \Rightarrow_{L} (*(obtener(c,e.familia))).id \leq e.cant}{\yluego}
% \repfunc{(\forall\param{}{c_{1},c_{2}}{Categoria}) (def?(c_{1},e.familia)) \land (def?(c_{2},e.familia)) \land c_{1}\neq c_{2}\Rightarrow_{L} \\
% (*(obtener(c_{1},e.familia))).id \neq (*(obtener(c_{2},e.familia))).id}{}
% \end{Rep}
% \newpage

% Abs
% .Abs(e) = sl: SL | 	gremios(sl) = armarConjDeGremios(e.gremios) yluego (paratodo g: gremio donde g pertenece gremios(sl)) aliados(g, sl) = armarConjDeAliados(e,g)

% armarConjDeAliados:
% 	recorre grupos, y para cada entrada que tenga un idGrupo igual al de g, agrega el gremio del mismo índice al resultado

\subsubsection{Funci\'on de Abstracci\'on}
\begin{ABS}{e}{estrS}{si}{sistemaLaboral}
\absfunc{gremios(si) \igobs obtenerConjuntoDeGremios(e)}{\yluego}
\absfunc{(\forall g: gremio) g \in gremios(si)\,\,\, entoncesLuego\,\,\, aliados(g, si) \igobs obtenerConjuntoDeAliados(e,g)}{}
\end{ABS}

\paragraph{Funciones auxiliares}
\paragraph*{}

\hspace*{1em}
\tadOperacion{obtenerConjuntoDeGremios}{}{conj(categoria)}{}
\tadAxioma{obtenerConjuntoDeGremios(cs)}{\IF vacia?(cs) THEN $\emptyset$ ELSE Ag((prim(cs)).categoria,obtenerConjuntoDeGremios(fin(cs))) FI}

% \hspace*{1em}
% \tadOperacion{predecesores}{arbolCategorias ac, Categoria c}{Conj(categoria)}{c $\in$ categorias(ac)}
% \tadAxioma{predecesores(ac,c)}{predecesoresAux(ac,categorias(ac),c)}

% \hspace*{1em}
% \tadOperacion{predecesoresAux}{arbolCategorias ac, Conj(Categoria) cc, Categoria c}{Conj(categoria)}{c $\in$ categorias(ac) $\land$ cc $\subseteq$ categorias(ac)}
% \tadAxioma{predecesoresAux(ac,cc,c)}{\IF $\emptyset$? cc THEN $\emptyset$ ELSE {\IF esSubCategoria(ac,dameUno(cc),c) THEN Ag(dameUno(cc),predecesoresAux(ac,sinUno(cc),c)) ELSE predecesoresAux(ac,sinUno(cc),c) FI} FI}