\begin{center}
\begin{tabular}{|l|} 
\hline
\\
Gremio \textbf{se representa con} estrG, donde estrG es: \\
\tupla{\\
\hspace*{4em}\param{}{empresas}{conj(empresa)},\hspace*{2em} \\
\hspace*{4em}\param{}{\#afiliados}{nat},\hspace*{2em} \\
\hspace*{4em}\param{}{idGremio}{nat},\hspace*{2em} \\
\hspace*{4em}\param{}{idGrupo}{nat} \\\hspace*{2em} } \\
\\
\hline
\end{tabular}
\end{center}

\subsubsection{Descripci\'on de los campos}

Los campos idGremio e idGrupo incorporan la información que permiten identificar al gremio y a sus aliados una vez que este se incorporó a un sistema laboral.

\subsubsection{Invariante de Representaci\'on}

\paragraph{Descripci\'on informal} Cualquier combinaci\'on de empresas, cantidad de afiliados, id de gremio e id de grupo de aliados representa una instancia v\'alida del tipo gremio.

\paragraph{Expresi\'on formal \\}
\RepFormalTrivial{estrG}{e}

\subsubsection{Funci\'on de Abstracci\'on}
\FunAbsExplicita{e}{estrG}{gremio}
{\\asignarIdGrupo(asignarIdGremio(nuevoGremio(e.empresas,e.\#afiliados), e.idGremio), e.idGrupo)}