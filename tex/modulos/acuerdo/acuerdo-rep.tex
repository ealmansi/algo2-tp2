\begin{center}
\begin{tabular}{|l|} 
\hline
\\
Acuerdo \textbf{se representa con} estrA, donde estrA es: \\
\tupla{\\
\hspace*{4em}\param{}{paritaria}{puntero(paritaria)},\hspace*{2em} \\
\hspace*{4em}\param{}{porcentaje}{nat},\hspace*{2em} \\
\hspace*{4em}\param{}{acuerdosPrevios}{nat} \\\hspace*{2em} } \\
\\
\hline
\end{tabular}
\end{center}

\subsubsection{Descripción de los campos}

\subsubsection{Invariante de Representaci\'on}

\paragraph{Descripción informal}

	El generador del TAD Acuerdo no provee restricciones, por lo cual cualquier combinación de paritaria, porcentaje y acuerdosPrevios puede representar una instancia válida (inclusive si porcentaje no se encuentra entre el piso y techo de la paritaria)

\paragraph{Expresión formal}
\RepFormalTrivial{estrA}{e}

\subsubsection{Funci\'on de Abstracci\'on}
\FunAbsExplicita{e}{estrA}{acuerdo}
{nuevoAcuerdo(*e.paritaria,e.porcentaje,e.acuerdosPrevios)}
\subsubsection{Funciones auxiliares}