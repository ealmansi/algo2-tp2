\begin{center}
\begin{tabular}{|l|} 
\hline
\\
Acuerdo \textbf{se representa con} estrA, donde estrA es: \\
\tupla{\\
\hspace*{4em}\param{}{paritaria}{paritaria},\hspace*{2em} \\
\hspace*{4em}\param{}{porcentaje}{nat},\hspace*{2em} \\
\hspace*{4em}\param{}{acuerdosPrevios}{nat} \\\hspace*{2em} } \\
\\
\hline
\end{tabular}
\end{center}

\subsubsection{Descripción de los campos}

	La paritaria a partir de la cual se generó el acuerdo se guarda en el campo homónimo, por si esta llegara a reabrirse.

	El campo porcentaje indica el valor de aumento acordado.

	El campo acuerdosPrevios cuenta la cantidad de acuerdos que realizó el gremio asociado.

\subsubsection{Invariante de Representaci\'on}

\paragraph{Descripción informal}

	El generador del TAD Acuerdo no provee restricciones, por lo cual cualquier combinación de paritaria, porcentaje y acuerdosPrevios puede representar una instancia válida (inclusive si porcentaje no se encuentra entre el piso y techo de la paritaria)

\paragraph{Expresión formal \\}
\RepFormalTrivial{estrA}{e}

\subsubsection{Funci\'on de Abstracci\'on}
\FunAbsExplicita{e}{estrA}{acuerdo}
{nuevoAcuerdo(e.paritaria,e.porcentaje,e.acuerdosPrevios)}