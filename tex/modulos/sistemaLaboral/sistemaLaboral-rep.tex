\begin{center}
\begin{tabular}{|l|} 
\hline
\\
SistemaLaboral \textbf{se representa con} estrS, donde estrS es: \\
\tupla{\\
\hspace*{4em}\param{}{gremios}{vector(puntero(gremio))},\hspace*{2em} \\
\hspace*{4em}\param{}{gruposDeAliados}{vector(idGrupo)} \\\hspace*{2em} } \\
\\
\hline
\end{tabular}
\end{center}

\subsubsection{Descripción de los campos}

\subsubsection{Invariante de Representaci\'on}

\paragraph{Descripción informal}

\begin{enumerate}
	\item Ninguna empresa aparece en más de un gremio.
	\item El id de cada gremio es igual a su índice en el vector gremios.
	\item Los vectores gremios y gruposDeAliados tienen igual longitud.
	\item Los valores del vector gruposDeAliados son consecutivos, comenzando en 0.
\end{enumerate}

\paragraph{Expresión formal}
\begin{RepFormal}{estrP}{e}
	\repItem{varias cosas}{}
\end{RepFormal}

\subsubsection{Funci\'on de Abstracci\'on}

\begin{FunAbsDescriptiva}{e}{estrS}{si}{sistemaLaboral}

	\absItem{gremios(si) \igobs obtenerConjuntoDeGremios(e)}{\yluego}

	\absItem{(\paratodof{g}{gremio}) g \en gremios(si) \entoncesL aliados(g, si) \igobs obtenerConjuntoDeAliados(e,g)}{}

\end{FunAbsDescriptiva}

\subsubsection{Funciones auxiliares}