\begin{center}
\begin{tabular}{|l|} 
\hline
\\
SistemaLaboral \textbf{se representa con} estrS, donde estrS es: \\
\tupla{\\
\hspace*{4em}\param{}{gremios}{conj(gremio)},\hspace*{2em} \\
\hspace*{4em}\param{}{gruposDeAliados}{vector(idGrupo)} \\\hspace*{2em} } \\
\\
\hline
\end{tabular}
\end{center}

\subsubsection{Descripción de los campos}

	El conjunto gremios guarda los gremios del sistema. A cada uno se le asigna durante la inicialización un id, tomando nat's consecutivos comenzando en 0.

	Por otro lado, dado que la relación de alianza entre gremios es simétrica y transitiva, se puede dividir al conjunto de gremios del sistema en grupos disjuntos de aliados.

	Utilizando esta propiedad, se representa la red de alianzas entre gremios etiquetando a cada grupo mediante un id de grupo, de forma tal que dos gremios serán aliados cuando sus números de grupo coincidan. El vector gruposDeAliados agrupa esta información, asignando a cada gremio un id de grupo, utilizando su id como índice en el vector. 

\subsubsection{Invariante de Representaci\'on}

\paragraph{Descripción informal}

\begin{enumerate}
	\item Ninguna empresa aparece en más de un gremio.
	\item Los ids de los gremios del conjunto son naturales consecutivos comenzando en 0.
	\item El cardinal del conjunto gremios y la longitud del vector gruposDeAliados coinciden.
	\item Los valores del vector gruposDeAliados son consecutivos, comenzando en 0.
\end{enumerate}

\paragraph{Expresión formal \\}
\begin{RepFormal}{estrP}{e}
	\repItem{varias cosas}{}
\end{RepFormal}

\subsubsection{Funci\'on de Abstracci\'on}

\begin{FunAbsDescriptiva}{e}{estrS}{si}{sistemaLaboral}

	\absItem{gremios(si) \igobs ObtenerGremios(e)}{\yluego}

	\absItem{(\paratodof{g}{gremio}) g \en gremios(si) \entoncesL aliados(g, si) \igobs obtenerConjuntoDeAliados(e,g)}{}

\end{FunAbsDescriptiva}

\subsubsection{Funciones auxiliares}