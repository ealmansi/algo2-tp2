\begin{center}
\begin{tabular}{|l|} 
\hline
\\
SistemaLaboral \textbf{se representa con} estrS, donde estrS es: \\
\tupla{\\
\hspace*{4em}\param{}{gremios}{vector(puntero(gremio))},\\
\hspace*{4em}\param{}{gruposDeAliados}{vector(idGrupo)} \\\hspace*{2em} } \\
\\
\hline
\end{tabular}
\end{center}

\subsubsection{Descripción de los campos}

\subsubsection{Invariante de Representaci\'on}

\paragraph{Descripción informal}

\begin{enumerate}
	\item Los gremios no comparten empresas.
	\item Los vectores \emph{gremios} y \emph{gruposDeAliados} tienen igual longitud (una entrada por cada gremio agregado).
	\item Los valores del vector \emph{gruposDeAliados} son consecutivos.
\end{enumerate}

\paragraph{Expresión formal}

\subsubsection{Funci\'on de Abstracci\'on}

\begin{ABS}{e}{estrS}{si}{sistemaLaboral}
\absfunc{gremios(si) \igobs obtenerConjuntoDeGremios(e)}{\yluego}
\absfunc{(\forall g: gremio) g \in gremios(si)\,\,\, entoncesLuego\,\,\, aliados(g, si) \igobs obtenerConjuntoDeAliados(e,g)}{}
\end{ABS}

\subsubsection{Funciones auxiliares}