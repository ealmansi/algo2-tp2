\begin{center}
\begin{tabular}{|l|} 
\hline
\\
SistemaLaboral \textbf{se representa con} estrS, donde estrS es: \\
\tupla{\\
\hspace*{4em}\param{}{gremios}{conj(gremio)},\hspace*{2em} \\
\hspace*{4em}\param{}{gruposDeAliados}{vector(idGrupo)} \\\hspace*{2em} } \\
\\
\hline
\end{tabular}
\end{center}

\subsubsection{Descripción de los campos}

	El conjunto gremios guarda los gremios del sistema. A cada uno se le asigna durante la inicialización un id, tomando nat's consecutivos comenzando en 0.

	Por otro lado, dado que la relación de alianza entre gremios es simétrica y transitiva, se puede dividir al conjunto de gremios del sistema en grupos disjuntos de aliados.

	Utilizando esta propiedad, se representa la red de alianzas entre gremios etiquetando a cada grupo mediante un id de grupo, de forma tal que dos gremios serán aliados cuando sus números de grupo coincidan. El vector gruposDeAliados agrupa esta información, asignando a cada gremio un id de grupo, utilizando su id como índice en el vector. 

\subsubsection{Invariante de Representaci\'on}

\paragraph{Descripción informal}

\begin{enumerate}
	\item Ninguna empresa aparece en más de un gremio.
	\item Los id's de los gremios del conjunto son naturales consecutivos no repetidos, comenzando en 0.
	\item El vector gruposDeAliados tiene una entrada para cada gremio, y ninguna m\'as.
	\item Los valores del vector gruposDeAliados son consecutivos, comenzando en 0.
\end{enumerate}

\paragraph{Expresión formal \\}
\begin{RepFormal}{estrP}{e}
	\repItem{
		(\paratodof{empresa}{\,em})\,((\lnot \exists \,g1,g2:gremio\, ) \, (g1 , g2 \in e.gremios \land em \in empresas(g1) \land em \in empresas(g2) )
	}{\ly}
	\repItem{
		(\paratodof{nat}{n})\, n \menor \#(e.gremios) \entonces ((\exists ! g : gremio)\, g \in e.gremios \land id(g) = n)
	}{\ly}
	\repItem{
		\#(e.gremios) = long(e.gruposDeAliados)
	}{\ly}
	\repItem{
		long(e.gruposDeAliados) > 0 \entonces esta?(0,e.gruposDeAliados) \ly ((\paratodof{nat}{n})\, (0 < n \ly esta?(n,e.gruposDeAliados)) \entonces esta?(n-1,e.gruposDeAliados))
	}{}
\end{RepFormal}

\subsubsection{Funci\'on de Abstracci\'on}

\begin{FunAbsDescriptiva}{e}{estrS}{si}{sistemaLaboral}

	\absItem{gremios(si) \igobs e.gremios}{\yluego}

	\absItem{(\paratodof{gremio}{g}) g \en gremios(si) \entoncesL aliados(g, si) \igobs obtenerConjuntoDeAliados(e.gruposDeAliados, e.gremios,g)}{}

\end{FunAbsDescriptiva}

\subsubsection{Funciones auxiliares}
\tadOperacion{obtenerConjuntoDeAliados}{indiceAliados/secu(nat), gremios/conj(gremio), g/gremio}{conj(gremio)}{}
\tadAxioma{obtenerConjuntoDeAliados}
{
	\textbf{if} $\emptyset$?(gremios) \textbf{then} \\
	\hspace*{20pt}$\emptyset$	\\
	\textbf{else} \\
		\hspace*{20pt}\textbf{if}(iesimo(id(dameUno(gremios), indiceAliados) \igobs iesimo(id(g), indiceAliados) \\
			\hspace*{40pt}dameUno(gremios) $\cup$ obtenerConjuntoDeAliados(sinUno(gremios), g) \\
		\hspace*{20pt}\textbf{else} \\
			\hspace*{40pt}obtenerConjuntoDeAliados(sinUno(gremios), g) \\
		\hspace*{20pt}\textbf{fi} \\
	\textbf{fi} - \{g\} \\
}