\begin{interfaz}{SistemaLaboral}
\begin{iparamformales}{sistemaLaboral}

\textbf{\large se explica con:} SistemaLaboral

\end{iparamformales}

\operacion{NuevoSistemaLaboral}
{\param{in/out}{gs}{conj(gremio)}}{sistemaLaboral}
{((\paratodof {gremio}{g1 ,g2} \in  gs) \, g1 \distinto g2 \entonces (empresas(g1) \cap empresas(g2)) = \emptyset) \ly \\ gsConsec = asignarIdsConsecutivos(gs)}
{res = nuevo(gsConsec) \ly esAlias(gremios(res), gsConsec)}
{\#gremios}
{Crea un sistema laboral a partir de un conjunto de gremios. No se deben modificar los ids de los gremios una vez ingresados en el sistema laboral}
{Los gremios se guardan por referencia}

\operacion{AliarGremios}
{\param{in/out}{sl}{sistemaLaboral}, \param{in}{gr1}{gremio}, \param{in}{gr2}{gremio}}{}
{gr1 \in gremios(sl) \ly gr2 \in gremios(sl) \ly gr1 \distinto gr2 \ly sl_0 = sl}
{idsGrupoActualizados(sl_0, sl, gr1, gr2)}
{\#gremios}
{Forma una alianza entre dos gremios (y entre sus aliados precedentes)}
{}

\operacion{ObtenerGremios}{\param{in}{sl}{sistemaLaboral}}{conj(gremio)}
{true}
{res = gremios(sl)}
{\#gremios}
{Devuelve los gremios del sistema}
{Los gremios se devuelven por copia}

\operacion{Obtener\#Grupos}
{\param{in}{sl}{sistemaLaboral}}{nat}
{true}
{res = maxIdGrupo(sl)}
{\#gremios}
{Cuenta la cantidad de grupos de aliados que existen en el sistema. }
{}

\end{interfaz}