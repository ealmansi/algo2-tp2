\begin{interfaz}{SistemaLaboral}
\begin{iparamformales}{sistemaLaboral}

\textbf{\large se explica con:} SistemaLaboral

\end{iparamformales}

\operacion{NuevoSistemaLaboral}
{\param{in/out}{gs}{conj(gremio)}}{sistemaLaboral}
{true}
{res = nuevo(gs) \ly esAlias(gremios(res), gs)}
{\#gremios}
{Crea un sistema laboral a partir de un conjunto de gremios}
{El conjunto de gremios se guarda por referencia}

\operacion{AliarGremios}
{\param{in/out}{sl}{sistemaLaboral}, \param{in}{gr1}{gremio}, \param{in}{gr2}{gremio}}{}
{gr1 \in gremios(sl) \ly gr2 \in gremios(sl) \ly gr1 \distinto gr2 }
{}
{\#gremios}
{Forma una alianza entre dos gremios (y entre sus aliados precedentes)}
{}

\operacion{ObtenerGremios}{\param{in}{sl}{sistemaLaboral}}{conj(gremio)}
{true}
{esAlias(res, gremios(sl))}
{1}
{Devuelve los gremios del sistema}
{El resultado se devuelve por referencia}

\operacion{Obtener\#GruposDeAliados}
{\param{in}{sl}{sistemaLaboral}}{nat}
{true}
{4}
{\#gremios}
{Cuenta la cantidad de grupos de aliados que existen en el sistema}
{}

\operacion{ObtenerIdGrupoDeAliados}
{\param{in}{sl}{sistemaLaboral}, \param{in}{gr}{gremio}}{idGrupo}
{gr \in gremios(sl)}
{4}
{1}
{Devuelve el id del grupo de aliados al que pertenece el gremio}
{}

\end{interfaz}