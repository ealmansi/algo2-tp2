\algoritmoI{NuevoSistemaLaboral}{}{vector(gremio)}
{
	\state res \asig Vacia()								\inlineC{1}
}
{1}
{Inicializa el sistema laboral}

\algoritmoI{AgregarGremio}{\param{in/out}{sl}{vector(gremio)}, \param{in}{conj(empresa)}{es},\param{in}{nat}{\#as}}{nat}
{	
	\state \var{nat}{idGremio} \asig Longitud(sl) 								\inlineC{1}
	\state \var{nat}{idGrupo} \asig MaximoIdGrupo(sl) 							\inlineC{\#gremios}
	\state

	\state \var{gremio}{gremioNuevo} \asig NuevoGremio(es, \#as) 				\inlineC{1}
	\state GuardarIdGremio(gremioNuevo, idGremio) 									\inlineC{1}
	\state GuardarIdGrupo(gremioNuevo, idGrupo) 										\inlineC{1}
	\state

	\state AgregarAtras(sl, gremioNuevo) 										\inlineC{\#gremios}
	\state

	\state res \asig idGremio 														\inlineC{1}
}
{\#gremios}
{Agrega un gremio al sistema laboral, conformado por una referencia al conjunto de empresas y la cantidad de afiliados recibidos, y devuelve el id del gremio registrado en el sistema.}

% \\hspace{10pt} El ciclo agrega \emph{\#gremios} elementos a un vector, lo cual puede no ser constante (la funci\'on \emph{f} del m\'odulo vector caracteriza el costo de la operación en función de la longitud del vector). Sin embargo, la operaci\'on tiene un costo amortizado de O(n) para n inserciones, por lo cual la funci\'on tiene un costo O(\#gremios). Los gremios se agregan por referencia en el vector, de forma tal que las sucesivas modificaciones en el id de grupo se realicen sobre los gremios originales. }

\algoritmoI{AliarGremios}{\param{in/out}{sl}{vector(gremio)}, \param{in}{idGr1}{nat}, \param{in}{idGr2}{nat}}{}
{
	\state \var{idGrupo}{id1} \asig ObtenerIdGrupo(sl[idGr1])						\inlineC{1}
	\state \var{idGrupo}{id2} \asig ObtenerIdGrupo(sl[idGr2])						\inlineC{1}
	\state 

	\If{id1 \distinto id2}												\inlineC{1}
% 		\state UnificarGruposDeAliados(sl, idGrupoG1, idGrupoG2)					\inlineC{\#gremios}
		
		\state \var{idGrupo}{idMenor} \asig Min(id1, id2)						\inlineC{1}
		\state \var{idGrupo}{idMayor} \asig Max(id1, id2)						\inlineC{1}
		\state

		\state \var{nat}{indice} \asig 0										\inlineC{1}

		\while{indice \menor Longitud(sl)}								\inlineC{1}
			\state

			\state \var{gremio}{gr} \asig sl[indice]					\inlineC{1}
			\state \var{idGrupo}{idG} \asig ObtenerIdGrupo(gr)					\inlineC{1}
			\state

			\If{idMayor \igual idG}												\inlineC{1}
				\state GuardarIdGrupo(gr, idMenor)								\inlineC{1}
			\Else \If{idMayor \menor idG}										\inlineC{1}
					\state GuardarIdGrupo(gr, idG - 1)							\inlineC{1}
				\endif
			\endif
			\state

			\state indice \asig indice + 1										\inlineC{1}
		\endwhile
	
		
	\endif
}
{\#gremios}
{ Asegura que todos los gremios aliados a uno u otro de los gremios que contraen la alianza pasen a tener el mismo id de grupo (manteniendo la propiedad de que los id's de grupo deben ser consecutivos, comenzando en 0). \\
Para lograr esto, a todos los gremios que pertenezcan al grupo de id mayor, se les asigna el menor. A la vez, todos los id's de grupo que sean mayores que ambos se decrementan en 1, garantizando que los id's de grupo sigan siendo consecutivos.
\hspace{10pt} Por otro lado, como las operaciones previas al ciclo, la evaluaci\'on de la guarda y su cuerpo se realizan en tiempo constante, y adem\'as el ciclo se ejecuta una vez por cada gremio del sistema, la funci\'on toma tiempo proporcional a \#gremios. }
% \hspace{10pt} Si la operaci\'on es no redundante (los gremios no eran aliados a\'un), el costo de esta operaci\'on queda determinado por la funci\'on UnificarGruposDeAliados. }

\algoritmoI{ObtenerGremios}{\param{in}{sl}{vector(gremio)}}{conj(gremio)}
{
	\state res \asig Vac\'io() \inlineC{1}
	\state \var{nat}{indice} \asig 0					\inlineC{1}
	\state 	
	\while{indice < \, Longitud(sl)}					\inlineC{1}
		\state 
		\state AgregarRapido(res, sl[indice])				\inlineC{1}
		\state
		\state indice \asig indice + 1					\inlineC{1}
	\endwhile

}
{\#gremios}
{ Se devuelve un conjunto con una copia de los gremios del sistema, lo cual toma tiempo lineal en la cantidad de gremios . }

\algoritmoI{ObtenerAliados}{\param{in}{sl}{vector(gremio)}, \param{in}{idGremio}{nat}}{conj(nat)}
{
 	\state res \asig Vac\'io() 															\inlineC{1}
 	\state \var{nat}{indice} \asig 0													\inlineC{1}
 	\state \var{nat}{idGrupoTarget} \asig ObtenerIdGrupo(sl[idGremio])					\inlineC{1}
 	\state 	
 	\while{(indice < \, Longitud(sl)) \ly (indice $\neq$ idGremio)}						\inlineC{1}
 		\state \If{ObtenerIdGrupo(sl[indice]) = idGrupoTarget}							\inlineC{1}
		  \state AgregarRapido(res, indice)												\inlineC{1}
		\endif
 		\state
 		\state indice \asig indice + 1													\inlineC{1}
 	\endwhile

}
{\#gremios}
{Acumula los id de gremio de los gremios aliados al requerido, lo cual en peor caso cuesta O(\#gremios)}


\algoritmoI{Obtener\#Grupos}{\param{in}{sl}{vector(gremio)}}{nat}
{
	\state res \asig MaximoIdGrupo(sl)						\inlineC{\#gremios}
}
{\#gremios}
{La cantidad de grupos de aliados en el sistema queda determinado por el m\'aximo id de grupo.}

\algoritmoI{MaximoIdGrupo}{\param{in}{sl}{vector(gremio)}}{nat}
{
	\state \var{nat}{maxId} \asig 0									\inlineC{1}

	\state
	\state \var{nat}{indice} \asig 0								\inlineC{1}
	\while{indice \menor Longitud(sl)}								\inlineC{1}
		\state
		
		\state maxId \asig Max(maxId, ObtenerIdGrupo(sl[indice]))			\inlineC{1}

		\state
		\state indice \asig indice + 1								\inlineC{1}
	\endwhile
	\state

	\state res \asig maxId											\inlineC{1}
}
{ \#gremios }
{ Las operaciones previas, la evaluaci\'on de la guarda y el cuerpo del ciclo se realizan en tiempo constante. Como se realiza un total de Longitud(sl) == \#gremios iteraciones, la funci\'on lleva tiempo lineal en funci\'on de \#gremios. }

\preAux{true}
\postAux{(\forall j : nat) ( j < long(sl)) \entoncesL idGrupo(sl[j]) \leq res}


\algoritmoI{Min}{\param{in}{id1}{idGrupo}, \param{in}{id2}{idGrupo}}{idGrupo}
{
	\If{id1 \menor id2}					\inlineC{1}
		\state res \asig id1			\inlineC{1}
	\Else
		\state res \asig id2			\inlineC{1}
	\endif
}
{1}
{}

\preAux{true}
\postAux{res \igobs min(id1, id2)}


\algoritmoI{Max}{\param{in}{id1}{idGrupo}, \param{in}{id2}{idGrupo}}{idGrupo}
{
	\If{id1 \menor id2}					\inlineC{1}
		\state res \asig id2			\inlineC{1}
	\Else
		\state res \asig id1			\inlineC{1}
	\endif
}
{1}
{}

\preAux{true}
\postAux{res \igobs max(id1, id2)}
