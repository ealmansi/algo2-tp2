\begin{interfaz}{Temporada}

\begin{iparamformales}{temporada}

\textbf{\large se explica con:} Temporada

\end{iparamformales}

\operacion{Iniciar}
{\param{in}{sl}{sistemaLaboral}}{temporada}
{true}
{res \igobs iniciar(sl)}
{1}
{Inicia una nueva temporada a partir de un sistema.}
{El sistema se guarda por referencia}

\operacion{AbrirParitaria}
{\param{in/out}{tmp}{temporada}, \param{in}{gr}{gremio}, \param{in}{p}{nat}, \param{in}{t}{nat}, \param{in}{es}{conj(empresa)}}{}
{gr \in gremios(tmp)\ly \neg enParitarias?(tpm, gr) \ly p \leq t \ly |es|*2 \leq |empresas(gr)| \ly es \subseteq empresas(gr)\ly tmp \igobs t0 }
{tmp \igobs abrirParitaria(t0, gr, p, t, es)  }
{1 si es la primera vez, de otra forma \# paritarias}
{Abre la paritaria de un gremio a partir de un piso, techo y un conjunto de empresas}
{La paritaria se agrega por copia}

\operacion{CerrarAcuerdo}
{\param{in/out}{tmp}{temporada}, \param{in}{gr}{gremio} , \param{in}{pcj}{nat} }{}
{enParitarias?(tmp, gr) \ly piso(paritaria(gr, tmp)) \leq pcj \leq tope(paritaria(gr, tmp) \ly tmp \igobs t0 }
{tmp = cerrarAcuerdo(t0, gr, pcj)}
{ \# acuerdosMenores + \# paritarias}
{Cierra un nuevo acuerdo y elimina la paritaria de ese gremio}
{El acuerdo se agrega por copia}

\operacion{Reabrir}
{\param{in/out}{tmp}{temporada}, \param{in}{gr}{gremio}}{}
{gr \in gremios(tmp)\ly \neg enParitarias?(tpm, gr) \ly \emptyset \neq  juntar(acuerdos(tmp, gr)) \ly tmp = t0 }
{tmp \igobs reabrir(t0, gr) }
{\# acuerdos}
{Reabre la paritaria de un gremio, eliminando su último acuerdo y agregando la paritaria}
{La paritaria se agrega por copia}

\operacion{Gremios}
{\param{in/out}{tmp}{temporada}}{conj(gremio)}
{true}
{res \igobs gremios(tmp) }
{ \# gremios }
{Devuelve un conjunto con los gremios del sistema de la temporada}
{Devuelve una referencia a el conjunto, que contiene copias de los gremios}

\operacion{EnParitarias}
{\param{in/out}{tmp}{temporada}, \param{in}{gr}{gremio}}{bool}
{true}
{ res \igobs enParitarias(tmp, gr) }
{\# paritarias}
{Indica si el gremio tiene su paritaria abierta en el momento}
{}

\operacion{GremiosNegociando}
{\param{in/out}{tmp}{temporada}}{conj(gremio)}
{true}
{res \igobs gremiosNegociando(tmp)}
{\# paritarias}
{Devuelve el conjunto de gremios que tienen su paritaria abierta }
{Devuelve una referencia a el conjunto, que contiene copias de los gremios}

\operacion{EmpresasNegociando}
{\param{in/out}{tmp}{temporada}}{conj(empresa)}
{true}
{res \igobs empresasNegociando(tmp)}
{falta implementar union}
{Devuelve el conjunto de empresas cuyos gremios tienen su paritaria abierta}
{falta implementar union}

\operacion{TrabajadoresNegociando}
{\param{in/out}{tmp}{temporada}}{nat}
{true}
{res \igobs trabajadoresNegociando(tmp)}
{\# paritarias}
{Devuelve el número de trabajadores cuyos gremios tienen su paritaria abierta}
{Se devuelve por copia}

\operacion{GremioConflictivo}
{\param{in/out}{tmp}{temporada}}{gremio}
{\emptyset \neq acuerdos(tmp)}
{res \igobs gremioConflictivo(tmp)}
{\# gremios}
{Devuelve el gremio que más veces reabrió su paritaria? creo que en la especificación devuelve el que más acuerdos anteriores tuvo, REVISAR }
{Devuelve una referencia al gremio}


\end{interfaz}