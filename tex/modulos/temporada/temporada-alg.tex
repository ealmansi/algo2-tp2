\algoritmoI{Iniciar}{\param{in}{sl}{sistemaLaboral}}{estrT}
{
	\state res.sistema \asig sl								\inlineC{1}
	\state res.paritariasAbiertas \asig Vacío()				\inlineC{1}
	\state res.acuerdosVigentesPorGrupo \asig Vacía()		\inlineC{1}
	\state res.\#aperturas \asig Vacía()					\inlineC{1}
	\state

	\state InicializarAcuerdosVigentesPorGrupo(res)			\inlineC{\#gremios}
	\state Inicializar\#Aperturas(res)						\inlineC{\#gremios}
}
{\#gremios}
{ Se guarda el sistema laboral, y se inicializan los contenedores de paritarias y acuerdos y los contadores de aperturas. Las asignaciones se realizan en tiempo constante y las inicializaciones en O(\#gremios), entonces la función es O(\#gremios). }

\algoritmoI{AbrirParitaria}{\param{in/out}{ tmp}{estrT}, \param{in}{gr}{gremio}, \param{in}{p}{nat}, \param{in}{t}{nat}, \param{in}{es}{conj(empresa)}}{}
{
	\If{0 \menor Obtener\#Aperturas(tmp, gr)}					\inlineC{1}
		\state RemoverAcuerdo(tmp, gr)							\inlineC{0}
	\endif
	\state

	\state AgregarParitaria(tmp, NuevaParitaria(gr, p, t))		\inlineC{0}
}
{0}
{\lipsum[\arabic{lipsumcounter}]\addtocounter{lipsumcounter}{1}}

\algoritmoI{CerrarAcuerdo}{\param{in/out}{ tmp}{estrT}, \param{in}{gr}{gremio} , \param{in}{pcj}{nat}}{}
{

	\state \var{itLista(acuerdo)}{itAcuerdos} \asig CrearIt(ObtenerAcuerdosDeAliados(tmp, gr))		\inlineC{0}
	\while{HaySiguiente(itAcuerdos) \lyl ObtenerPorcentaje(Siguiente(itAcuerdos)) \menor pcj}		\inlineC{0}
		\state

		\state AgregarParitaria(tmp, ObtenerParitaria(Siguiente(itAcuerdos)))		\inlineC{0}

		\state EliminarSiguiente(itAcuerdos)										\inlineC{0}

		\state
		\state Avanzar(itAcuerdos)													\inlineC{0}
	\endwhile
	\state

	\state \var{paritaria}{pa} \asig RemoverParitaria(tmp, gr)				\inlineC{0}
	\state AgregarComoSiguiente(itAcuerdos, NuevoAcuerdo(pa, pcj))			\inlineC{0}
}
{0}
{\lipsum[\arabic{lipsumcounter}]\addtocounter{lipsumcounter}{1}}

\algoritmoI{Reabrir}{\param{in/out}{ tmp}{estrT},  \param{in}{gr}{gremio}}{}
{
	\state \var{paritaria}{pa} \asig RemoverAcuerdo(tmp, gr)			\inlineC{0}

	\state AgregarParitaria(tmp, pa)									\inlineC{0}
}
{0}
{\lipsum[\arabic{lipsumcounter}]\addtocounter{lipsumcounter}{1}}

\algoritmoI{Gremios}{\param{in}{tmp}{estrT}}{conj(gremio)}
{
	\state res \asig ObtenerGremios(tmp.sl)
}
{0}
{\lipsum[\arabic{lipsumcounter}]\addtocounter{lipsumcounter}{1}}

\algoritmoI{EnParitarias}{\param{in}{tmp}{estrT}, \param{in}{gr}{gremio}}{bool}
{
	\state \var{itConj(paritaria)}{itParitarias} \asig CrearIt(tmp.paritariasAbiertas)		\inlineC{0}
	\while{HaySiguiente(itParitarias) \lyl ObtenerGremio(Siguiente(itParitarias)) \distinto gr}		\inlineC{0}
		\state
		\state Avanzar(itParitarias)				\inlineC{0}
	\endwhile
	\state

	\state res \asig HaySiguiente(itParitarias)		\inlineC{0}
}
{0}
{\lipsum[\arabic{lipsumcounter}]\addtocounter{lipsumcounter}{1}}

\algoritmoI{GremiosNegociando}{\param{in}{tmp}{estrT}}{conj(gremio)}
{
	\state res \asig Vacío()		\inlineC{0}
	\state

	\state \var{itConj(paritaria)}{itParitarias} \asig CrearIt(tmp.paritariasAbiertas)		\inlineC{0}
	\while{HaySiguiente(itParitarias)}										\inlineC{0}
		\state
		\state Agregar(res, ObtenerGremio(Siguiente(itParitarias)))		\inlineC{0}
		
		\state
		\state Avanzar(itParitarias)										\inlineC{0}
	\endwhile
}
{0}
{\lipsum[\arabic{lipsumcounter}]\addtocounter{lipsumcounter}{1}}

\algoritmoI{EmpresasNegociando}{\param{in}{tmp}{estrT}}{conj(empresa)}
{
	\state res \asig Vacío()		\inlineC{0}
	\state

	\state \var{itConj(paritaria)}{itParitarias} \asig CrearIt(tmp.paritariasAbiertas)		\inlineC{0}
	\while{HaySiguiente(itParitarias)}						\inlineC{0}
		\state
		\state conj(empresa) empresasParitaria \asig ObtenerEmpresas(ObtenerGremio(Siguiente(itParitarias)))		\inlineC{0}
		\state res \asig Union(res, empresasParitaria)		\inlineC{0}

		\state
		\state Avanzar(itParitarias)						\inlineC{0}
	\endwhile
}
{0}
{\lipsum[\arabic{lipsumcounter}]\addtocounter{lipsumcounter}{1}}

\algoritmoI{TrabajadoresNegociando}{\param{in}{tmp}{estrT}}{nat}
{
	\state res \asig 0		\inlineC{0}
	\state

	\state \var{itConj(paritaria)}{itParitarias} \asig CrearIt(tmp.paritariasAbiertas)		\inlineC{0}
	\while{HaySiguiente(itParitarias)}					\inlineC{0}
		\state

		\state \var{nat}{\#afiliadosParitaria} \asig Obtener\#Afiliados(ObtenerGremio(Siguiente(itParitarias)))		\inlineC{0}
		\state res \asig res + \#afiliadosParitaria		\inlineC{0}

		\state
		\state Avanzar(itParitarias)					\inlineC{0}
	\endwhile
}
{0}
{\lipsum[\arabic{lipsumcounter}]\addtocounter{lipsumcounter}{1}}

\algoritmoI{GremioConflictivo}{\param{in}{tmp}{estrT}}{gremio}
{
	\state \var{itConj(gremio)}{itGremios} \asig CrearIt(Gremios(tmp))						\inlineC{0}
	\state

	\state \var{gremio}{gremioConflictivo} \asig Siguiente(itGremios)				\inlineC{0}
	\state \var{nat}{maxAperturas} \asig Obtener\#Aperturas(tmp, gremioConflictivo)		\inlineC{0}
	\state Avanzar(itGremios)																\inlineC{0}
	\state

	\while{HaySiguiente(itGremios)}																\inlineC{0}
		\state

		\state \var{nat}{\#aperturas} \asig Obtener\#Aperturas(tmp, Siguiente(itGremios))		\inlineC{0}

		\If{maxAperturas \menor \#aperturas}							\inlineC{0}

			\state maxAperturas \asig \#aperturas						\inlineC{0}
			\state gremioConflictivo \asig Siguiente(itGremios)			\inlineC{0}
		\endif

		\state
		\state Avanzar(itGremios)										\inlineC{0}
	\endwhile
	\state

	\state res \asig gremioConflictivo									\inlineC{0}
}
{0}
{\lipsum[\arabic{lipsumcounter}]\addtocounter{lipsumcounter}{1}}

\algoritmoI{InicializarAcuerdosVigentesPorGrupo}{\param{in/out}{ tmp}{estrT}}{}
{
	\state \var{nat}{indice} \asig 0									\inlineC{1}
	\while{indice \menor Obtener\#GruposDeAliados(tmp.sistema)}			\inlineC{1}
		\state

		\state tmp.acuerdosVigentesPorGrupo[indice] \asig Vacía()		\inlineC{1}

		\state
		\state indice \asig indice + 1									\inlineC{1}
	\endwhile

}
{\#gremios}
{ La operación inicial, la evaluación de la guarda, y el cuerpo del ciclo se realizan en tiempo constante. Como el ciclo se repite tantas veces como \#grupos\_de\_aliados, la función es O(\#grupos\_de\_aliados). Pero a su vez, la cantidad de grupos de aliados es igual a la cantidad de gremios en el peor caso, entonces la función es O(\#gremios). }

\algoritmoI{Inicializar\#Aperturas}{\param{in/out}{ tmp}{estrT}}{}
{
	\state \var{nat}{indice} \asig 0					\inlineC{1}
	\while{indice \menor Cardinal(Gremios(tmp))}		\inlineC{1}
		\state

		\state tmp.\#aperturas[indice] \asig 0			\inlineC{1}

		\state
		\state indice \asig indice + 1					\inlineC{1}
	\endwhile
}
{\#gremios}
{ La operación inicial, la evaluación de la guarda, y el cuerpo del ciclo se realizan en tiempo constante. Como el ciclo se repite tantas veces como \#gremios, la función es O(\#gremios). }

\algoritmoI{AgregarParitaria}{\param{in/out}{ tmp}{estrT}, \param{in}{pa}{paritaria}}{}
{
	\state Agregar(tmp.paritariasAbiertas, pa)		\inlineC{0}
}
{0}
{\lipsum[\arabic{lipsumcounter}]\addtocounter{lipsumcounter}{1}}

\algoritmoI{RemoverParitaria}{\param{in/out}{ tmp}{estrT}, \param{in}{gr}{gremio}}{paritaria}
{
	\state \var{itConj(paritaria)}{itParitarias} \asig CrearIt(tmp.paritariasAbiertas)		\inlineC{0}
	\while{ObtenerGremio(Siguiente(itParitarias)) \distinto gr}		\inlineC{0}
		\state
		\state Avanzar(itParitarias)								\inlineC{0}
	\endwhile
	\state

	\state res \asig Siguiente(itParitarias)						\inlineC{0}

	\state EliminarSiguiente(itParitarias)							\inlineC{0}
}
{0}
{\lipsum[\arabic{lipsumcounter}]\addtocounter{lipsumcounter}{1}}

\algoritmoI{RemoverAcuerdo}{\param{in/out}{ tmp}{estrT}, \param{in}{gr}{gremio}}{paritaria}
{
	\state \var{itLista(acuerdo)}{itAcuerdos} \asig CrearIt(ObtenerAcuerdosDeAliados(tmp, gr))		\inlineC{1}
	\while{ObtenerGremio(Siguiente(itAcuerdos)) \distinto gr}		\inlineC{0}
		\state
		\state Avanzar(itAcuerdos)  								\inlineC{1}
	\endwhile
	\state

	\state res \asig ObtenerParitaria(Siguiente(itAcuerdos))		\inlineC{1}

	\state EliminarSiguiente(itAcuerdos)							\inlineC{1}
}
{0}
{\lipsum[\arabic{lipsumcounter}]\addtocounter{lipsumcounter}{1}}

\algoritmoI{ObtenerAcuerdosDeAliados}{\param{in}{tmp}{estrT}, \param{in}{gr}{gremio}}{lista(acuerdo)}
{
	\state \var{idGrupo}{id} \asig ObtenerIdGrupoAliados(tmp.sistema, gr)		\inlineC{0}

	\state res \asig tmp.acuerdosVigentesPorGrupo[id]							\inlineC{0}
}
{0}
{\lipsum[\arabic{lipsumcounter}]\addtocounter{lipsumcounter}{1}}

\algoritmoI{Obtener\#Aperturas}{\param{in}{tmp}{estrT}, \param{in}{gr}{gremio}}{nat}
{
	\state \var{idGremio}{id} \asig ObtenerId(gr)			\inlineC{1}

	\state res \asig tmp.\#aperturas[id]					\inlineC{1}
}
{1}
{ El tiempo de lectura de la i-ésima posición de un vector es constante, entonces la función permite conocer la cantidad de aperturas de un gremio en O(1). }