\algoritmoI{Iniciar}{\param{in}{sl}{sistemaLaboral}}{estrT}
{
	\state res.sistema \asig sl								\inlineC{1}
	\state res.paritariasAbiertas \asig Vac\'io()				\inlineC{1}
	\state res.acuerdosPorGrupo \asig Vac\'ia()		\inlineC{1}
	\state res.cantAcuerdosPrevios \asig Vac\'ia()					\inlineC{1}
	\state

	\state InicializarAcuerdosPorGrupo(res)			\inlineC{\#gremios}
	\state InicializarCantAcuerdosPrevios(res)						\inlineC{\#gremios}
}
{\#gremios}
{ Se guarda el sistema laboral, y se inicializan los contenedores de paritarias y acuerdos y los contadores de acuerdos previos. Las asignaciones se realizan en tiempo constante y las inicializaciones en O(\#gremios), entonces la funci\'on es O(\#gremios). }

\algoritmoI{AbrirParitaria}{\param{in/out}{ tmp}{estrT}, \param{in}{gr}{gremio}, \param{in}{p}{nat}, \param{in}{t}{nat}, \param{in}{es}{conj(empresa)}}{}
{
	\If{0 \menor ObtenerCantAcuerdosPrevios(tmp, gr)}				\inlineC{1}
		\state RemoverAcuerdo(tmp, gr)							\inlineC{\#acuerdos\_menores}
	\endif
	\state

	\state AgregarParitaria(tmp, NuevaParitaria(gr, p, t))		\inlineC{1}
}
{\#acuerdos\_menores}
{ Cuando la paritaria de un gremio abre por primera vez, no existe acuerdo previo en el sistema y su contador de acuerdos previos es nulo; en ese caso, la operaci\'on se realiza en O(1). Para el caso general, la complejidad temporal de la operaci\'on queda determinada por la funci\'on RemoverAcuerdo. }

\algoritmoI{CerrarAcuerdo}{\param{in/out}{ tmp}{estrT}, \param{in}{gr}{gremio} , \param{in}{pcj}{nat}}{}
{

	\state \var{itLista(acuerdo)}{itAcuerdos} \asig CrearIt(ObtenerAcuerdosDeAliados(tmp, gr))		\inlineC{1}
	\while{HaySiguiente(itAcuerdos) \lyl ObtenerPorcentaje(Siguiente(itAcuerdos)) \menor pcj}		\inlineC{1}
		\state

		\state AgregarParitaria(tmp, ObtenerParitaria(Siguiente(itAcuerdos)))		\inlineC{1}

		\state EliminarSiguiente(itAcuerdos)										\inlineC{1}

		\state
		\state Avanzar(itAcuerdos)													\inlineC{1}
	\endwhile
	\state

	\state \var{paritaria}{pa} \asig RemoverParitaria(tmp, gr)											\inlineC{\#paritarias\_abiertas}
	\state IncrementarCantAcuerdosPrevios(tmp, gr)														\inlineC{1}
	\state AgregarComoSiguiente(itAcuerdos, NuevoAcuerdo(pa, pcj, ObtenerCantAcuerdosPrevios(tmp, gr)))	\inlineC{1}
}
{\#acuerdos\_menores + \#paritarias\_abiertas}
{La funci\'on ObtenerAcuerdosDeAliados devuelve la lista de los aliados del gremio afectado en tiempo constante; luego, la creaci\'on del iterador para esta lista tambi\'en se realiza en O(1). 

\hspace{10pt} La evaluaci\'on de la guarda y el cuerpo del ciclo se realizan en tiempo constante, ya que todas estas funciones se realizan en O(1): HaySiguiente, ObtenerPorcentaje, Siguiente, AgregarParitaria, ObtenerParitaria, EliminarSiguiente, Avanzar, \lyl, \menor.

\hspace{10pt} La lista de acuerdos de aliados se itera secuencialmente hasta encontrar un acuerdo con un porcentaje de aumento mayor o igual al logrado en el nuevo acuerdo. Como la lista se encuentra ordenada de menor a mayor seg\'un el valor del aumento, esto significa que se realizar\'an exactamente \#acuerdos\_menores iteraciones. En cada una de ellas, se reabre la paritaria de un acuerdo de menor aumento, efectivizando el comportamiento autom\'atico.

\hspace{10pt} La operaci\'on RemoverParitaria es lineal en la cantidad de paritarias abiertas, y por \'ultimo, agregar un acuerdo a la lista mediante el iterador se realiza en tiempo constante. Por lo tanto, la funci\'on tiene una complejidad temporal de O(\#acuerdos\_menores + \#paritarias\_abiertas).  }

\algoritmoI{Reabrir}{\param{in/out}{ tmp}{estrT},  \param{in}{gr}{gremio}}{}
{
	\state \var{paritaria}{pa} \asig RemoverAcuerdo(tmp, gr)			\inlineC{\#acuerdos\_menores}

	\state AgregarParitaria(tmp, pa)									\inlineC{1}
}
{\#acuerdos\_menores}
{ Elimina el acuerdo vigente del gremio en cuesti\'on, y agrega la paritaria que estaba guardada ah\'i. La segunda operaci\'on se realiza en tiempo constante, por lo que RemoverAcuerdo marca el costo temporal de la funci\'on.}

\algoritmoI{Gremios}{\param{in}{tmp}{estrT}}{conj(gremio)}
{
	\state res \asig ObtenerGremios(tmp.sl) \inlineC{\#gremios}
}
{\#gremios}
{ Se devuelve un conjunto con una copia de los gremios de la temporada, lo cual toma tiempo lineal en la cantidad de gremios . }

\algoritmoI{EnParitarias}{\param{in}{tmp}{estrT}, \param{in}{gr}{gremio}}{bool}
{
	\state \var{itConj(paritaria)}{itParitarias} \asig CrearIt(tmp.paritariasAbiertas)		\inlineC{1}
	\while{HaySiguiente(itParitarias) \lyl \\ \hspace{20pt} ObtenerId(ObtenerGremio(Siguiente(itParitarias))) \distinto ObtenerId(gr)}		\inlineC{1}
		\state
		\state Avanzar(itParitarias)				\inlineC{1}
	\endwhile
	\state

	\state res \asig HaySiguiente(itParitarias)		\inlineC{1}
}
{\#paritarias\_abiertas}
{ Itera sobre todas las paritarias abiertas hasta encontrar la del gremio, por lo que en el peor caso repite el ciclo \emph{\#paritarias\_abiertas} veces. }

\algoritmoI{GremiosNegociando}{\param{in}{tmp}{estrT}}{conj(gremio)}
{
	\state res \asig Vac\'io()		\inlineC{1}
	\state

	\state \var{itConj(paritaria)}{itParitarias} \asig CrearIt(tmp.paritariasAbiertas)		\inlineC{1}
	\while{HaySiguiente(itParitarias)}										\inlineC{1}
		\state
		\state AgregarRapido(res, ObtenerGremio(Siguiente(itParitarias)))			\inlineC{1}
		
		\state
		\state Avanzar(itParitarias)										\inlineC{1}
	\endwhile
}
{\#paritarias\_abiertas}
{ Para cada paritaria abierta, agrega el gremio asociado al resultado. Como la operaci\'on de agregado en el conjunto se realiza por copia, la complejidad temporal es de O(\#paritarias\_abiertas * costo de copiar un gremio). Como el gremio guarda el conjunto de empresas por referencia, la copia del gremio consiste en copiar una referencia y copiar un nat (la cantidad de afiliados), por lo que se puede realizar en tiempo constante. }

\algoritmoI{EmpresasNegociando}{\param{in}{tmp}{estrT}}{conj(empresa)}
{
	\state res \asig Vac\'io()		\inlineC{1}
	\state

	\state \var{itConj(paritaria)}{itParitarias} \asig CrearIt(tmp.paritariasAbiertas)		\inlineC{1}
	\while{HaySiguiente(itParitarias)}						\inlineC{1}
		\state
		\state conj(empresa) empresasParitaria \asig ObtenerEmpresas(ObtenerGremio(Siguiente(itParitarias)))		\inlineC{1}
		\state
		\state \var{itConj(empresa)}{itEmpresas} \asig CrearIt(empresasParitaria)	\inlineC{1}
		\while{HaySiguiente(itEmpresas)}	\inlineC{1}
			\state
			\state AgregarRapido(res, Siguiente(itEmpresas))	\inlineC{1}
			
			\state
			\state Avanzar(itEmpresas)	\inlineC{1}
		\endwhile

		\state
		\state Avanzar(itParitarias)						\inlineC{1}
	\endwhile
}
{ \ensuremath{\#paritarias\_abiertas * max\_cant\_empresas * max\_long\_empresa }}
{ \#paritarias\_abiertas es la cantidad de paritarias abiertas de la temporada, max\_cant\_empresas es la m\'axima cantidad de empresas que tiene un gremio (entre aquellos con paritarias abiertas), y max\_long\_empresa es el nombre de empresa m\'as largo de todas ellas. Dado que el m\'odulo conjunto agrega los elementos por copia, el costo de la funci\'on es el costo de copiar cada empresa de cada gremio con una paritaria abierta. }

\algoritmoI{TrabajadoresNegociando}{\param{in}{tmp}{estrT}}{nat}
{
	\state res \asig 0		\inlineC{1}
	\state

	\state \var{itConj(paritaria)}{itParitarias} \asig CrearIt(tmp.paritariasAbiertas)		\inlineC{1}
	\while{HaySiguiente(itParitarias)}					\inlineC{1}
		\state

		\state \var{nat}{\#afiliadosParitaria} \asig Obtener\#Afiliados(ObtenerGremio(Siguiente(itParitarias)))		\inlineC{1}
		\state res \asig res + \#afiliadosParitaria		\inlineC{1}

		\state
		\state Avanzar(itParitarias)					\inlineC{1}
	\endwhile
}
{ \#paritarias\_abiertas }
{ Todas las operaciones se realizan en tiempo constante; como se realiza una iteraci\'on por cada paritaria abierta, la funci\'on es O(\#paritarias\_abiertas). }

\algoritmoI{GremioConflictivo}{\param{in}{tmp}{estrT}}{gremio}
{
	\state \var{itConj(gremio)}{itGremios} \asig CrearIt(Gremios(tmp))							\inlineC{1}
	\state

	\state \var{gremio}{gremioConflictivo} \asig Siguiente(itGremios)							\inlineC{1}
	\state \var{nat}{mCantAcuerdosPrevios} \asig ObtenerCantAcuerdosPrevios(tmp, gremioConflictivo)				\inlineC{1}
	\state Avanzar(itGremios)																	\inlineC{1}
	\state

	\while{HaySiguiente(itGremios)}																\inlineC{1}
		\state

		\state \var{nat}{cantAcuerdosPrevios} \asig ObtenerCantAcuerdosPrevios(tmp, Siguiente(itGremios))		\inlineC{1}

		\If{mCantAcuerdosPrevios \menor cantAcuerdosPrevios}							\inlineC{1}

			\state mCantAcuerdosPrevios \asig cantAcuerdosPrevios						\inlineC{1}
			\state gremioConflictivo \asig Siguiente(itGremios)			\inlineC{1}
		\endif

		\state
		\state Avanzar(itGremios)										\inlineC{1}
	\endwhile
	\state

	\state res \asig gremioConflictivo									\inlineC{1}
}
{\#gremios}
{Obtiene el gremio con cantidad m\'axima de acuerdos previos. Para esto se efectua una busqueda lineal sobre el conjunto de gremios, por lo que esta funci\'on posee un costo temporal O(\#gremios).}

\algoritmoI{InicializarAcuerdosPorGrupo}{\param{in/out}{ tmp}{estrT}}{}
{
	\state \var{nat}{indice} \asig 0									\inlineC{1}
	\while{indice \menor Obtener\#GruposDeAliados(tmp.sistema)}			\inlineC{1}
		\state

		\state tmp.acuerdosPorGrupo[indice] \asig Vac\'ia()		\inlineC{1}

		\state
		\state indice \asig indice + 1									\inlineC{1}
	\endwhile

}
{\#gremios}
{ La operaci\'on inicial, la evaluaci\'on de la guarda, y el cuerpo del ciclo se realizan en tiempo constante. Como el ciclo se repite tantas veces como \#grupos\_de\_aliados, la funci\'on es O(\#grupos\_de\_aliados). Pero a su vez, la cantidad de grupos de aliados es igual a la cantidad de gremios en el peor caso, entonces la funci\'on es O(\#gremios). }

Pre : \ensuremath{tmp0=tmp}
\\
Post : \ensuremath{ tienenListasVacias(tmp.acuerdosPorGrupo) \land (tmp.sistema \igobs tmp0.sistema) \land (tmp.paritariasAbiertas \igobs tmp0.paritariasAbiertas) \land (tmp.\#acuerdosPrevios \igobs tmp0.\#acuerdosPrevios)}

\algoritmoI{InicializarCantAcuerdosPrevios}{\param{in/out}{ tmp}{estrT}}{}
{
	\state \var{nat}{indice} \asig 0					\inlineC{1}
	\while{indice \menor Cardinal(Gremios(tmp))}		\inlineC{1}
		\state

		\state tmp.cantAcuerdosPrevios[indice] \asig 0			\inlineC{1}

		\state
		\state indice \asig indice + 1					\inlineC{1}
	\endwhile
}
{\#gremios}
{ La operaci\'on inicial, la evaluaci\'on de la guarda, y el cuerpo del ciclo se realizan en tiempo constante. Como el ciclo se repite tantas veces como \#gremios, la funci\'on es O(\#gremios). }

Pre : \ensuremath{tmp0=tmp}
\\
Post : \ensuremath{ tieneTodoCero(tmp.\#acuerdosPrevios) \land (tmp.sistema \igobs tmp0.sistema) \land (tmp.paritariasAbiertas \igobs tmp0.paritariasAbiertas) \land (tmp.acuerdosPorGrupo \igobs tmp0.acuerdosPorGrupo)}

\algoritmoI{AgregarParitaria}{\param{in/out}{ tmp}{estrT}, \param{in}{pa}{paritaria}}{}
{
	\state AgregarRapido(tmp.paritariasAbiertas, pa)		\inlineC{1}
}
{1}
{ La funci\'on debe utilizarse con la certeza de que la paritaria que se va a agregar no se encuentra abierta, para no degenerar el conjunto paritariasAbiertas. La paritaria se agrega al conjunto por copia, la cual se realiza en O(copy(pa)), pero esto es O(1) ya que solo requiere copiar la referencia al gremio, y dos valores de tipo nat. }

Pre : \ensuremath{tmp0=tmp}
\\
Post : \ensuremath{ (tmp.paritariasAbiertas=Ag(pa, tmp0.paritariasAbiertas)) \land (tmp.sistema \igobs tmp0.sistema) \land (tmp.\#acuerdosPrevios \igobs tmp0.\#acuerdosPrevios) \land (tmp.acuerdosPorGrupo \igobs tmp0.acuerdosPorGrupo)}

\algoritmoI{RemoverParitaria}{\param{in/out}{ tmp}{estrT}, \param{in}{gr}{gremio}}{paritaria}
{
	\state \var{itConj(paritaria)}{itParitarias} \asig CrearIt(tmp.paritariasAbiertas)		\inlineC{1}
	\while{ObtenerId(ObtenerGremio(Siguiente(itParitarias))) \distinto ObtenerId(gr)}		\inlineC{1}
		\state
		\state Avanzar(itParitarias)								\inlineC{1}
	\endwhile
	\state

	\state res \asig Siguiente(itParitarias)						\inlineC{1}

	\state EliminarSiguiente(itParitarias)							\inlineC{1}
}
{\#paritarias\_abiertas}
{Se iteran secuencialmente las paritarias abiertas hasta encontrar la requerida,tomando tiempo lineal en el peor caso cuando la paritaria esta al final en el orden del iterador
. Las operaciones con el iterador toman todas un costo temporal constante. Por lo tanto la funci\'on tiene un costo temporal tiempo lineal
}

Pre : \ensuremath{tmp0=tmp \land (0 \leq idgremio(gr) < long(tmp.paritariasAbiertas)) \yluego buscarParitaria(idgremio(gr), tmp.paritariasAbiertas) \neq \emptyset}
\\
Post : \ensuremath{ (res \igobs buscarParitaria(idgremio(gr), tmp0.paritariasAbiertas)) \land (tmp.paritariasAbiertas=tmp0.paritariasAbiertas - \{ res \}) \land (tmp.sistema \igobs tmp0.sistema) \land (tmp.\#acuerdosPrevios \igobs tmp0.\#acuerdosPrevios) \land (tmp.acuerdosPorGrupo \igobs tmp0.acuerdosPorGrupo)}

\algoritmoI{RemoverAcuerdo}{\param{in/out}{ tmp}{estrT}, \param{in}{gr}{gremio}}{paritaria}
{
	\state \var{itLista(acuerdo)}{itAcuerdos} \asig CrearIt(ObtenerAcuerdosDeAliados(tmp, gr))		\inlineC{1}
	\while{ObtenerId(ObtenerGremio(Siguiente(itAcuerdos))) \distinto ObtenerId(gr)}		\inlineC{1}
		\state
		\state Avanzar(itAcuerdos)  								\inlineC{1}
	\endwhile
	\state

	\state res \asig ObtenerParitaria(Siguiente(itAcuerdos))		\inlineC{1}

	\state EliminarSiguiente(itAcuerdos)							\inlineC{1}
}
{\#acuerdos\_menores}
{Se iteran secuencialmente los acuerdos de los aliados del gremio hasta encontrar el propio. Como los acuerdos se encuentran ordenados por porcentaje, se realizar\'an \emph{\#acuerdos\_menores} iteraciones hasta encontrar el acuerdo buscado, por lo cual la funci\'on tiene un costo de O(\#acuerdos\_menores). }

Pre : \ensuremath{ (tmp0 \igobs tmp) \land (gr \in gremios(tmp)) \yluego tieneAcuerdo(tmp, gr))}
\\
Post : \ensuremath{ (tmp \igobs removerAcuerdo(tmp0)) \land (res \igobs paritaria(acuerdo(gr, tmp0))) }

\algoritmoI{ObtenerAcuerdosDeAliados}{\param{in}{tmp}{estrT}, \param{in}{gr}{gremio}}{lista(acuerdo)}
{
	\state \var{idGrupo}{id} \asig ObtenerIdGrupo(gr)		\inlineC{1}

	\state res \asig tmp.acuerdosPorGrupo[id]							\inlineC{1}
}
{1}
{ El m\'odulo SistemaLaboral permite obtener el id de grupo de un gremio en tiempo constante, y luego solo es necesario leer la posici\'on correspondiente en el vector acuerdosPorGrupo. }

Pre : \ensuremath{tmp0=tmp \land (0 \leq idgrupo(gr) < long(tmp.acuerdosPorGrupo))}
\\
Post : \ensuremath{ (res \igobs tmp.acuerdosPorGrupo_{idgrupo(gr)})}

\algoritmoI{IncrementarCantAcuerdosPrevios}{\param{in/out}{tmp}{estrT}, \param{in}{gr}{gremio}}{}
{
	\state \var{idGremio}{id} \asig ObtenerId(gr)											\inlineC{1}

	\state tmp.cantAcuerdosPrevios[id] \asig tmp.cantAcuerdosPrevios[id] + 1					\inlineC{1}
}
{1}
{ El tiempo de lectura de la i-\'esima posici\'on de un vector es constante, entonces la funci\'on incrementa la cantidad de acuerdos previos de un gremio en O(1). }

Pre : \ensuremath{tmp0=tmp \land (0 \leq idgremio(gr) < long(tmp.\#acuerdosPrevios))}
\\
Post : \ensuremath{ (tmp.\#acuerdosPrevios_{idgremio(gr)} \igobs tmp0.\#acuerdosPrevios_{idgremio(gr)}-1) \land ((\forall i:nat) (0 \leq idgremio(gr) < long(tmp.\#acuerdosPrevios) \land i \neq idgremio(gr)) \entoncesL tmp0.\#acuerdosPrevios_i) \land (tmp.sistema \igobs tmp0.sistema) \land (tmp.paritariasAbiertas \igobs tmp0.paritariasAbiertas) \land (tmp.acuerdosPorGrupo \igobs tmp0.acuerdosPorGrupo)}

\algoritmoI{ObtenerCantAcuerdosPrevios}{\param{in}{tmp}{estrT}, \param{in}{gr}{gremio}}{nat}
{
	\state \var{idGremio}{id} \asig ObtenerId(gr)			\inlineC{1}

	\state res \asig tmp.cantAcuerdosPrevios[id]					\inlineC{1}
}
{1}
{ El tiempo de lectura de la i-\'esima posici\'on de un vector es constante, entonces la funci\'on permite conocer la cantidad de acuerdos previos de un gremio en O(1). }

Pre : \ensuremath{(0 \leq idgremio(gr) < long(tmp.\#acuerdosPrevios))}
\\
Post : \ensuremath{ res \igobs tmp.\#acuerdosPrevios_{idgremio(gr)}}

\subsection{Funciones Auxiliares de Pre-Post}

\tadOperacion{tienenListasVacias}{secu(secu(acuerdo))/s}{bool}{}
\tadOperacion{tieneTodoCero}{secu(nat)/s}{bool}{}
\tadOperacion{buscarParitaria}{nat/idGremio, conj(paritaria) cp}{paritaria}{}
\tadOperacion{obtenerAcuerdoDeAliados}{secu(secu(acuerdo))/s, nat idGr}{secu(acuerdo)}{}
\tadOperacion{removerAcuerdo}{Acuerdo ac, secu(secu(acuerdo))/ss}{secu(acuerdo)}{}
\vspace{20px}
\tadAxioma{tienenListasVacias(s)}{if (vacia?(s)) then true else (prim(s) \igobs <>) \ly tienenListasVacias(fin(s)) fi}
\tadAxioma{tieneTodoCero(s)}{if (vacia?(s)) then true else (prim(s) \igobs 0) \ly tieneTodoCero(fin(s) fi)}
\tadAxioma{buscarParitaria(idG, cp)}{if (idgremio(gremio(dameUno(cp))) \igobs idG) then dameUno(cp) else buscarParitaria(idGr, sinUno(cp)) fi)}
\tadAxioma{obtenerAcuerdoDeAliados(s, idGr)}{if (idGr \igobs 1) then prim(s) else obtenerAcuerdoDeAliados(fin(s), idGr-1) fi}
\tadAxioma{removerAcuerdo(ac, ss)}{
if(vacia?(ss)) then
  <> 
else
  if(!esta(ac, prim(ss))) then
    prim(ss) \puntito removerAcuerdo(ac, fin(ss))
  else
    sacar(ac,prim(ss)) \puntito removerAcuerdo(ac, fin(ss))
  fi
fi
}

\tadAxioma{sacar(ac, s)}{
if(vacia?(s)) then
  <> 
else
  if(prim(s) \igobs ac) then
    sacar(ac, fin(s))
  else
    prim(s) \puntito sacar(ac, fin(s))
  fi
fi
}