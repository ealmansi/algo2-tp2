\begin{center}
\begin{tabular}{|l|} 
\hline
\\
Temporada \textbf{se representa con} estrT, donde estrT es: \\
\tupla{\\
\hspace*{4em}\param{}{sistema}{sistemaLaboral},\hspace*{2em} \\
\hspace*{4em}\param{}{paritariasAbiertas}{conj(paritaria)},\hspace*{2em} \\
\hspace*{4em}\param{}{acuerdosVigentesPorGrupo}{vector(lista(acuerdo))},\hspace*{2em} \\
\hspace*{4em}\param{}{\#acuerdosPrevios}{vector(nat)} \\\hspace*{2em} } \\
\\
\hline
\end{tabular}
\end{center}

\subsubsection{Descripción de los campos}

\subsubsection{Invariante de Representaci\'on}

\paragraph{Descripción informal \\ \\}
\begin{enumerate}

	\item Todo gremio del sistema tiene una (y solo una) paritaria abierta, o un (y solo un) acuerdo vigente, o ninguna de las dos cosas.
	\item Toda paritaria en paritariasAbiertas pertence a algun gremio del sistema.
	\item Todo acuerdo en acuerdosVigentesPorGrupo pertence a algun gremio del sistema.
	\item AcuerdosVigentesPorGrupo tiene una lista para cada grupo de aliados del sistema, y ninguna más.
	\item Dos gremios alojan sus acuerdos en la misma lista de acuerdosVigentesPorGrupo si, y solo si, son aliados.
	\item Cada lista de acuerdosVigentesPorGrupo está ordenada segun porcentaje.
	\item Todo acuerdo tiene un porcentaje que está entre el piso y el techo de su paritaria.
	\item \#acuerdosPrevios tiene una entrada para cada gremio del sistema, y ninguna más.
	\item Los valores en el vector \#acuerdosPrevios son consistentes con los valores de acuerdosPrevios de los acuerdos.

\end{enumerate}

\paragraph{Expresión formal \\}
\begin{RepFormal}{estrT}{e}
	\repItem{muchas cosas}{}
\end{RepFormal}

\subsubsection{Funci\'on de Abstracci\'on}

\begin{FunAbsDescriptiva}{e}{estrS}{si}{sistemaLaboral}

	\absItem{the great choclo}{\yluego}

\end{FunAbsDescriptiva}

\subsubsection{Funciones auxiliares}
