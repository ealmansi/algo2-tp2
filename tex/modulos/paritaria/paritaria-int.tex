\begin{interfaz}{Paritaria}
\begin{iparamformales}{paritaria}

\comment{\funcion{Copiar}{\param{in}{m}{$\alpha$}}{$\alpha$}
{true}
{\igres m}
{copy(m)}
{}
}

\textbf{\large se explica con:} Paritaria

\end{iparamformales}

\operacion{NuevaParitaria}{\param{in}{gr}{gremio}, \param{in}{ps}{nat}, \param{in}{tp}{nat}}{paritaria}{ps \menorigual tp}{res \igobs nuevaParitaria (gr, ps, tp)  \ly esAlias(gremio(res), gr) }{1}{Crea una paritaria a partir de un gremio, un valor piso y un valor tope.}{}
\operacion{ObtenerGremio}{\param{in}{pa}{paritaria}}{gremio)}{true}{esAlias(res, gremio(pa)  }{1}{Devuelve el gremio de la paritaria.}{}
\operacion{ObtenerPiso}{\param{in}{pa}{paritaria}}{nat}{true}{res \igobs piso (pa)}{1}{Devuelve el valor piso de la paritaria.}{}
\operacion{ObtenerTope}{\param{in}{pa}{paritaria}}{nat}{true}{res \igobs tope (pa)}{1}{Devuelve el valor tope de la paritaria.}{}

\end{interfaz}