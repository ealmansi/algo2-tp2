\begin{interfaz}{Paritaria}

\begin{iparamformales}{paritaria}

\textbf{\large se explica con:} Paritaria

\end{iparamformales}

\operacion{NuevaParitaria}
{\param{in}{gr}{gremio}, \param{in}{ps}{nat}, \param{in}{tp}{nat}}{paritaria}
{ps \menorigual tp}
{res \igobs nuevaParitaria (gr, ps, tp)  \ly esAlias(gremio(res), gr) }
{1}
{Crea una paritaria a partir del gremio al que corresponde, y los valores de piso y tope}
{El gremio se guarda por referencia dentro de la paritaria}

\operacion{ObtenerGremio}
{\param{in}{pa}{paritaria}}{gremio)}
{true}
{esAlias(res, gremio(pa)  }
{1}
{Devuelve el gremio de la paritaria}
{El resultado se devuelve por referencia}

\operacion{ObtenerPiso}
{\param{in}{pa}{paritaria}}{nat}
{true}
{res \igobs piso (pa)}
{1}
{Devuelve el valor del piso de la paritaria}
{}

\operacion{ObtenerTope}
{\param{in}{pa}{paritaria}}{nat}
{true}
{res \igobs tope (pa)}
{1}
{Devuelve el valor del tope de la paritaria}
{}

\end{interfaz}