\algoritmoI{NuevaParitaria}
{\param{in}{gr}{gremio}, \param{in}{ps}{nat}, \param{in}{tp}{nat}}{estrP}
{
	\state res.gremio \asig gr										\inlineC{1}
	\state res.piso \asig ps										\inlineC{1}
	\state res.tope \asig tp										\inlineC{1}
}
{1}
{Crea una paritaria a partir de un gremio, un valor piso y un valor tope.}

\algoritmoI{ObtenerGremio}
{\param{in}{pa}{estrP}}{gremio}
{
	\state res \asig pa.gremio										\inlineC{1}
}
{1}
{Devuelve el gremio de la paritaria.}

\algoritmoI{ObtenerPiso}
{\param{in}{pa}{estrP}}{nat}
{
	\state res \asig pa.piso										\inlineC{1}
}
{1}
{Devuelve el valor "piso" de la paritaria.}

\algoritmoI{ObtenerTope}
{\param{in}{pa}{estrP}}{nat}
{
	\state res \asig pa.tope										\inlineC{1}
}
{1}
{Devuelve el valor "tope" de la paritaria.}