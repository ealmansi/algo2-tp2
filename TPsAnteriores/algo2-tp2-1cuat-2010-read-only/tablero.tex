\section{Tablero}
\subsection{Interfaz}

\begin{interfaz}{Tablero}

\textbf{par�metros formales}

\generos{$\mu$}

\operacion{Copiar}{\param{in}{m}{$\mu$}}{$\mu$}
{true}
{\res \igobs m}
{copy(m)}
{}

\textbf{se explica con} Tablero($\mu$)

\generos{tablero($\mu$)}

\textbf{Operaciones}

\operacion{Inicial}{}{tablero}
{true}
{\res = Inicial()}
{1}
{no hay}

\operacion{Contener}{\param{inout}{t}{tablero($\mu$)}, \param{in}{c}{casilla}, \param{in}{m_i}{$\mu$}, \param{in}{m_v}{$\mu$}}{}
{t = t_0 \wedge c \in Casillas(t) \wedge m_v \notin Salidas(t, c)}
{t = Contener(t_0, c, m_i, m_v)}
{f(ProxCasilla($t_0$)) + copy($m_i$) + copy($m_v$)}
{no hay}

\operacion{Agregar}{\param{inout}{t}{tablero($\mu$)}, \param{in}{c}{casilla}, \param{in}{m_i}{$\mu$}, \param{in}{m_v}{$\mu$}}{}
{t = t_0 \wedge c \in Casillas(t) \wedge m_v \notin Salidas(t, c)}
{t = Agregar(t{_0}, c, m_i, m_v)}
{f(ProxCasilla(t)) + f(ProxContinente(t)) + copy($m_i$) + copy($m_v$)}
{no hay}

\operacion{Movilizar}{\param{inout}{t}{tablero($\mu$)}, \param{in}{c_1}{casilla}, \param{in}{c_2}{casilla}, \param{in}{m}{$\mu$}}{}
{t = t_0 \wedge \{c_1, c_2\} \subseteq Casillas(t) \wedge c_1 \neq c_2 \wedge \neg Conectadas(t, c_1, c_2) \wedge m \notin Salidas(t, c_1)}
{t = Movilizar(t_0, c_1, c_2, m)}
{copy(m)}
{no hay}

\operacion{Conectar}{\param{inout}{t}{tablero($\mu$)}, \param{in}{c_1}{casilla}, \param{in}{c_2}{casilla}, \param{in}{m_i}{$\mu$}, \param{in}{m_v}{$\mu$}}{}
{t = t_0 \wedge \{c_1, c_2\} \subseteq Casillas(t) \wedge c_1 \neq c_2 \wedge \neg Conectadas(t, c_1, c_2) \wedge m_i \notin Salidas(t, c_1) \wedge m_v \notin Salidas(t, c_2)}
{t = Conectar(t_0, c_1, c_2, m_i, m_v)}
{copy($m_i$) + copy($m_v$)}
{no hay}

\operacion{Casillas}{\param{in}{t}{tablero($\mu$)}}{conj(casilla)}
{true}
{\res = Casillas(t)}
{1}
{alias(\res, Casillas(t))}

\operacion{Continente}{\param{in}{t}{tablero($\mu$)}, \param{in}{c}{casilla}}{continente}
{c \in Casillas(t)}
{\res = Continente(t, c)}
{1}
{alias(\res, Continente(t, c))}

\operacion{Origenes}{\param{in}{t}{tablero($\mu$)}, \param{in}{c}{casilla}, \param{in}{m}{$\mu$}}{conj(casilla)}
{c \in Casillas(t)}
{\res = Origenes(t, c, m)}
{\#Origenes(t, c, m)}
{no hay}

\operacion{Continentes}{\param{in}{t}{tablero($\mu$)}}{conj(continente)}
{true}
{\res = Continentes(t)}
{1}
{alias(\res, Continentes(t))}

\operacion{Conectadas}{\param{in}{t}{tablero($\mu$)}, \param{in}{c_1}{casilla}, \param{in}{c_2}{casilla}}{bool}
{\{c_1, c_2\} \subseteq Casillas(t)}
{\res = Conectadas(t_0, c_1, c_2)}
{\#Casillas(t)}
{no hay}

\operacion{Salidas}{\param{in}{t}{tablero($\mu$)}, \param{in}{c}{casilla}}{conj($\mu$)}
{c \in Casillas(t)}
{\res = Salidas(t, c)}
{1}
{alias(\res, Salidas(t, c))}

\operacion{OrigenesR}{\param{in}{t}{tablero($\mu$)}, \param{in}{c}{casilla}, \param{in}{m}{$\mu$}}{conj(casilla)}
{c \in Casillas(t)}
{\res = OrigenesR(t, c, m)}
{\#Origenes(t, c, m)}
{no hay}

\operacion{CasillasDe}{\param{in}{t}{tablero($\mu$)}, \param{in}{cc}{continente}}{conj(casilla)}
{cc \in Continentes(t)}
{\res = CasillasDe(t, cc)}
{1}
{alias(\res, CasillasDe(t, cc))}

\operacion{Destinos}{\param{in}{t}{tablero($\mu$)}, \param{in}{c}{casilla}}{itConj(conj(tupla(casilla, $\mu$))}
{c \in Casillas(t)}
{\res = CrearIt(Destinos(t, c))}
{1}
{alias(\res, Destinos(t, c))}

\operacion{ProxCasilla}{\param{in}{t}{tablero($\mu$)}}{casilla}
{true}
{\res = \#Casillas(t) + 1}
{1}
{no hay}

\operacion{ProxContinente}{\param{in}{t}{tablero($\mu$)}}{continente}
{true}
{\res = \#Continentes(t) + 1}
{1}
{no hay}

\end{interfaz}


\subsection{Representaci�n}

\subsubsection{Estructura de Representaci�n}

tablero($\mu$) se representa con \textbf{t}

\vspace{12pt}

donde \textbf{t} es tupla(casillas: dicc(casilla, datosCasilla), continentes: dicc(continente, conj(casilla)))

\vspace{12pt}

donde \textbf{datosCasilla} es tupla(salidas: conj($\mu$), \\
origenes: conj(tupla(casilla:casilla, movimiento:$\mu$)), \\
destinos: conj(tupla(casilla:casilla, movimiento:$\mu$)), \\
continente: continente)

\vspace{12pt}

\subsubsection{Invariante de Representaci�n}

1. Para todo datosCasilla dc en t.datosCasilla, dc.continente est� contenido en las claves de t.continentes.

2. Para todo conjunto de casillas cs en t.continentes, cs est� inclu�do en claves de t.casillas.

3. Toda casilla en claves de t.casillas pertenece a uno solo de los conjuntos de casillas en t.continentes.

4. Para todo datosCasilla dc en t.datosCasilla, para toda tupla tup en dc.origenes: tup.casilla pertenece a t.casillas

5. Para todo datosCasilla dc en t.datosCasilla, para toda tupla tup en dc.destinos: tup.casilla pertenece a t.casillas

6. Para todo datosCasilla dc en t.datosCasilla, para toda tupla tupd en dc.destinos, tupd.movimiento pertenece a dc.salidas

7. Para todo datosCasilla dc en t.datosCasilla, Para todo movimiento m en t.salidas, existe una tupla en dc.destinos que tiene a m como segunda componente

8. Cada casilla c2 que aparece en las tuplas de destinos de c1, tiene una tupla en origenes donde la primer componente es c1 y la segunda componente es el mismo movimiento que el de destino de c1

9. Cada casilla c2 que aparece en las tuplas de origenes de c1, tiene una tupla en destinos donde la primer componente es c1 y la segunda componente es el mismo movimiento que el de origen de c1

\vspace{12pt}

\tadOperacion{Rep}{t}{bool}{}

\tadAxioma{Rep(t)}{true $\Leftrightarrow$}

($\forall$ c:casilla)(def?(c, t.casillas) $\Rightarrow$ \\
\hspace*{20pt}($\forall$ dc:datosCasilla)(dc = Obtener(c, t.casillas) $\Rightarrow$ \\
	\hspace*{40pt}(dc.continente $\in$ claves(t.continentes)) \textbf{[1]} $\wedge$ \\
	\hspace*{40pt}(($\forall$ cc:continente)(def?(cc, t.continentes) $\Leftrightarrow$ \\
		\hspace*{60pt}(($\forall$ cs:conj(casilla))(cs = Obtener(cc, t.continentes) $\Leftrightarrow$ (cs $\in$ claves(t.casillas)) \textbf{[2]} $\wedge$ (($\exists$! cs2:conj(casilla))(cs2 = cs $\wedge$ c $\in$ cs2)) \textbf{[3]})))) $\wedge$ \\
	\hspace*{40pt}(($\forall$ tup:tupla(casilla, $\mu$))((tup $\in$ dc.origenes $\Rightarrow$ \\
		\hspace*{60pt}tup.casilla $\in$ claves(t.casillas)) \textbf{[4]} $\wedge$ \\
		\hspace*{60pt}(tup $\in$ dc.destinos $\Rightarrow$ tup.casilla $\in$ claves(t.casillas) $\wedge$ tup.movimiento $\in$ dc.salidas) \textbf{[5 y 6]})) $\wedge$ \\
	\hspace*{40pt}(($\forall$ ms:$\mu$)(m $\in$ dc.salidas $\Rightarrow$ (($\exists$ tupd:tupla(casilla,$\mu$))(tupd $\in$ dc.destinos $\wedge$ tupd.movimiento = m)))) \textbf{[7]} $\wedge$ \\
	\hspace*{40pt}(($\forall$ c2:casilla)(def?(c2, t.casillas) $\wedge$ c2 $\neq$ c $\Rightarrow$ \\
		\hspace*{60pt}(($\forall$ dc2:datosCasilla)(dc2 = Obtener(c2, t.casillas) $\Rightarrow$ \\
			\hspace*{80pt}(($\forall$ tup:tupla(casilla, $\mu$))((tup $\in$ dc.destinos $\Rightarrow$ \\
				\hspace*{100pt}(($\exists$ tupo:tupla(casilla, $\mu$))(tupo $\in$ dc2.origenes $\wedge$ \\
					\hspace*{120pt}tupo.casilla = c $\wedge$ tupo.movimiento = tup.movimiento))) \textbf{[8]} $\wedge$ \\
				\hspace*{100pt}(tup $\in$ dc.origenes $\Rightarrow$ \\
					\hspace*{120pt}(($\exists$ tupd:tupla(casilla, $\mu$))(tupd $\in$ dc2.destinos $\wedge$ \\
						\hspace*{140pt}tupd.casilla = c $\wedge$ tupd.movimiento = tup.movimiento)) \textbf{[9]} \\
				\hspace*{100pt}) \\
			\hspace*{80pt})) \\
		\hspace*{60pt})) \\
	\hspace*{40pt})) \\
\hspace*{20pt}) \\
)


\subsubsection{Funci�n de Abstracci�n}

\tadOperacion{Abs}{e:t}{tablero($\mu$)}{Rep(e)}

\tadAxioma{Abs(e)}{t:tablero($\mu$) / \\
Casillas(t) = claves(e.casillas) $\wedge$ \\
(($\forall$ c:casilla)(c $\in$ Casillas(t) $\Rightarrow$ \\
\hspace*{20pt}($\exists$ dc:datosCasilla)(dc = (Obtener(c, e.casillas)) $\wedge$ \\
	\hspace*{40pt}Continente(t, c) = dc.continente $\wedge$ \\
	\hspace*{40pt}(($\forall$ m:$\mu$)(Origenes(t, c, m) = OrigenesAux(dc.origenes, m)))) \\
))
}

\vspace{10pt}

\tadOperacion{OrigenesAux}{conj(tupla$<$casilla,$\mu$$>$) \, os, $\mu$ \, m}{conj(casilla)}{}

\tadAxioma{OrigenesAux(os, m)}{\LIF($\emptyset$?(os)) \LTHEN \\
							   \hspace*{20pt}$\emptyset$ \\
							   \LELSE \\
							   \hspace*{20pt}\LIF ($\Pi_2$(damUno(os)) = m) \LTHEN \\
							   		\hspace*{40pt}Ag($\Pi_1$(dameUno(os)), OrigenesAux(sinoUno(os), m)) \\
							   \hspace*{20pt}\LELSE \\
							   		\hspace*{40pt}OrigenesAux(sinUno(os), m) \\
							   \hspace*{20pt}\LFI \\
							   \LFI}


\subsection{Algoritmos}

\begin{algorithm}{iInicial}{}{res: tablero}
\VAR continenteCero : continente \\
continenteCero \= 0 \\
\VAR casillaCero : casilla \\
casillaCero \= 0 \\
\VAR diccCasillas : dicc(casilla, datosCasilla) \\
diccCasillas \= Vacio() \comentario{O(1)} \\
\VAR datosCasilla : datosCasilla \\
datosCasilla \= <Vacio(), Vacio(), Vacio(), continenteCero> \comentario{O(1)} \\
DefinirRapido(casillaCero, datosCasilla, diccCasillas) \comentario{O(1)} \\
\VAR diccContinentes : dicc(continente, conj(casilla)) \\
diccContinentes \= Vacio() \comentario{O(1)} \\
\VAR conjCasillas : conj(casilla) \\
conjCasillas \= Vacio() \comentario{O(1)} \\
AgregarRapido(conjCasillas, casillaCero) \comentario{O(1)} \\
DefinirRapido(continenteCero, conjCasillas, diccContinentes) \comentario{O(\#(conjCasillas)), en este caso hay una sola casilla} \\
res \= <diccCasillas,diccContinentes> \comentario{O(1) por referencia}
\end{algorithm}

\begin{algorithm}{iContener}{\param{inout}{t}{tablero($\mu$)}, \param{in}{c}{casilla}, \param{in}{m_i}{$\mu$}, \param{in}{m_v}{$\mu$}}{}
\VAR datosCasillaExistente : datosCasilla \\
datosCasillaExistente \= Obtener(c, t.casillas) \comentario{O(1) por referencia}\\
\VAR continente : continente \\
continente \= datosCasillaExistente.continente \comentario{O(1) por referencia} \\
\VAR nuevaCasilla : casilla \\
nuevaCasilla \= iProxCasilla(t) \comentario{O(1)}\\
\VAR datosNuevaCasilla : datosCasilla \\
\comentario{armo tupla de datos de la nueva casilla} \\
datosNuevaCasilla \= <AgregarRapido(\emptyset, copy(m_i)), \comentario{O(copy($m_i$))} \\
\hspace*{1.1em} AgregarRapido(\emptyset, <c, copy(m_v>)), \comentario{O(copy($m_v$))} \\
\hspace*{1.1em} AgregarRapido(\emptyset, <c, copy(m_i)>), continente>  \comentario{O(copy($m_i$)} \\
DefinirRapido(nuevaCasilla, datosNuevaCasilla, t.casillas) \comentario{O(copy(nuevaCasilla) + copy(datosNuevaCasilla))} \\
\comentario{, en este caso es f(ProxCasilla(t)) por la expansion del vector en caso de que est� completo.}  \\
\VAR conjCasillasContinente : conj(casilla) \\
conjCasillasContinente \= Obtener(continente, t.continentes) \comentario{O(1) por referencia} \\
AgregarRapido(conjCasillasContinente, nuevaCasilla) \comentario{O(1)}
\comentario{agrego casilla nueva como origen de c,$m_i$} \\
AgregarRapido(datosCasillaExistente.origenes, <nuevaCasilla, copy(m_i)>) \comentario{O(copy($m_i$))} \\
\comentario{agrego casilla nueva como destino de c,$m_v$} \\
AgregarRapido(datosCasillaExistente.destinos, <nuevaCasilla, copy(m_v>)) \comentario{O(copy($m_v$))}
\end{algorithm}

\begin{algorithm}{iAgregar}{\param{inout}{t}{tablero($\mu$)}, \param{in}{c}{casilla}, \param{in}{m_i}{$\mu$}, \param{in}{m_v}{$\mu$}}{}
\VAR datosCasillaExistente : datosCasilla \\
datosCasillaExistente \= Obtener(c, t.casillas) \comentario{O(1) por referencia} \\
\VAR continente : continente \\
continente \= datosCasillaExistente.continente \comentario{O(1) por referencia} \\
\VAR nuevoContinente : continente \\
nuevoContinente \= iProxContinente(t) \comentario{O(1)} \\
\VAR nuevaCasilla : casilla \\
nuevaCasilla \= iProxCasilla(t) \comentario{O(1)} \\
\VAR datosNuevaCasilla : datosCasilla \\
\comentario{armo tupla de datos de la nueva casilla} \\
datosNuevaCasilla \= <AgregarRapido(\emptyset, copy(m_i)), \\
\hspace*{1.1em} AgregarRapido(\emptyset, <c, copy(m_v)>), \comentario{O(copy($m_v$))} \\
\hspace*{1.1em} AgregarRapido(\emptyset, <c, copy(m_i)>), nuevoContinente> \comentario{O(copy($m_i$) + copy($m_v$))} \\
DefinirRapido(nuevaCasilla, datosNuevaCasilla, t.casillas) \comentario{O(copy($m_i$) + copy($m_v$)) + f(ProxCasilla(t))} \\
\comentario{, en este caso es f(ProxCasilla(t)) por la expansion del vector en caso de que est� completo.}  \\
\VAR conjCasillasContinente : conj(casilla) \\
conjCasillasContinente \= AgregarRapido(nuevaCasilla, \emptyset) \comentario{O(1)} \\
DefinirRapido(nuevoContinente, conjCasillasContinente, t.continentes)\\
\comentario{O(f(ProxContinente(t)) por la expansion del vector en caso de que est� completo.} \\
\comentario{agrego casilla nueva como origen de c,$m_i$} \\
AgregarRapido(datosCasillaExistente.origenes, <nuevaCasilla, copy(m_i)>) \comentario{O(copy($m_i$))}
\comentario{agrego casilla nueva como destino de c,$m_v$} \\
AgregarRapido(datosCasillaExistente.destinos, <nuevaCasilla, copy(m_v)>) \comentario{O(copy($m_v$))}
\end{algorithm}

\begin{algorithm}{iMovilizar}{\param{inout}{t}{tablero($\mu$)}, \param{in}{c_1}{casilla}, \param{in}{c_2}{casilla}, \param{in}{m}{$\mu$}}{}
\VAR datosCasilla1 : datosCasilla \\
datosCasilla1 \= Obtener(c_1, t.casillas) \comentario{O(1) por referencia} \\
AgregarRapido(datosCasilla1.salidas, m) \comentario{O(1)} \\
AgregarRapido(datosCasilla1.destinos, <c_2, copy(m)>) \comentario{O(copy(m))} \\
\VAR datosCasilla2 : datosCasilla \\
datosCasilla2 \= Obtener(c_2, t.casillas) \comentario{O(1) por referencia} \\
\comentario{agrego casilla nueva como origen de $c_2$,m} \\
AgregarRapido(datosCasilla2.origenes, <c_1, copy(m)>) \comentario{O(copy(m))}
\end{algorithm}

\begin{algorithm}{iConectar}{\param{inout}{t}{tablero($\mu$)}, \param{in}{c_1}{casilla}, \param{in}{c_2}{casilla}, \param{in}{m_i}{$\mu$}, \param{in}{m_v}{$\mu$}}{}
\VAR datosCasilla1 : datosCasilla \\
datosCasilla1 \= Obtener(c_1, t.casillas) \comentario{O(1) por referencia} \\
AgregarRapido(datosCasilla1.salidas, m_i) \comentario{O(1)} \\
\comentario{agrego $c_2$ como origen de $c_1$,$m_v$} \\
AgregarRapido(datosCasilla1.origenes, <c_2, copy(m_v)>) \comentario{O(copy($m_v$))} \\
AgregarRapido(datosCasilla1.destinos, <c_2, copy(m_i)>) \comentario{O(copy($m_i$))} \\
\VAR datosCasilla2 : datosCasilla \\
datosCasilla2 \= Obtener(c_2, t.casillas) \comentario{O(1) por referencia} \\
AgregarRapido(datosCasilla2.salidas, copy(m_v)) \comentario{O(1)} \\
\comentario{agrego casilla nueva como origen de $c_2$,$m_i$} \\
AgregarRapido(datosCasilla2.origenes, <c_1, copy(m_i)>) \comentario{O(copy($m_i$))} \\
AgregarRapido(datosCasilla2.destinos, <c_1, copy(m_v)>) \comentario{O(copy($m_v$))}
\end{algorithm}

\begin{algorithm}{iCasillas}{\param{in}{t}{tablero($\mu$)}}{\res: conj(casilla)}
\res \= claves(t.casillas) \comentario{O(1) por referencia} \\
return \, \res
\end{algorithm}

\begin{algorithm}{iContinente}{\param{in}{t}{tablero($\mu$)}, \param{in}{c}{casilla}}{\res: continente}
\VAR datosCasilla : datosCasilla \\
datosCasilla \= Obtener(c, t.casillas) \comentario{O(1) por referencia} \\
\res \= datosCasilla.continente \comentario{O(1) por referencia} \\
return \, \res
\end{algorithm}

\begin{algorithm}{iOrigenes}{\param{in}{t}{tablero($\mu$)}, \param{in}{c}{casilla}, \param{in}{m}{$\mu$}}{\res: conj(casilla)}
\res \= \emptyset \\
\VAR datosCasilla : datosCasilla \\
datosCasilla \= Obtener(c, t.casillas) \comentario{O(1) por referencia} \\
\VAR it : itConj \\
it \= CrearIt(datosCasilla.origenes) \comentario {O(1) creo el iterador de conj}\\
\VAR cm : tupla(casilla:casilla, movimiento:movimiento) \\
\begin{WHILE}{HaySiguiente(it)}
	cm \= Siguiente(it) \\
	\begin{IF}{cm.movimiento = m}
		AgregarRapido(cm.casilla, \res)\\
	\end{IF}
	Avanzar(it)
\end{WHILE} \\
return \; res \\
\comentario {recorro todas los origenes O(\#Origenes(t, c, m))}
\end{algorithm}

\begin{algorithm}{iContinentes}{\param{in}{t}{tablero($\mu$)}}{\res: conj(continente)}
\res \= claves(t.continentes) \comentario{O(1) por referencia} \\
return \, \res
\end{algorithm}

\begin{algorithm}{iConectadas}{\param{in}{t}{tablero($\mu$)}, \param{in}{c_1}{casilla}, \param{in}{c_2}{casilla}}{\res: bool}
\comentario{c2 es destino de c1?} \\
\VAR datosCasilla : datosCasilla \\
datosCasilla \= Obtener(c1, t.casillas) \comentario{O(1) por referencia} \\
\VAR it : itConj \\
it \= CrearIt(datosCasilla.destinos) \comentario {O(1) creo el iterador de conj}\\
\VAR cm : tupla(casilla:casilla, movimiento:movimiento) \\
res \= true \\
\begin{WHILE}{\res \wedge HaySiguiente(it)}
	cm \= Siguiente(it) \\
	\res \= (cm.casilla = c2) \\
	Avanzar(it)\\
\end{WHILE}
\comentario{recorro todos los destinos de c1 O(\#Casillas(t))} \\
\comentario{si c2 no es destino de c1, c1 es destino de c2?} \\
\begin{IF}{\neg \res}
	datosCasilla \= Obtener(c2, t.casillas) \comentario{O(1) por referencia} \\
	it \= CrearIt(datosCasilla.destinos) \comentario {O(1) creo el iterador de conj}\\
	res \= true \\
	\begin{WHILE}{\res \wedge HaySiguiente(it)}
		cm \= Siguiente(it) \\
		\res \= cm.casilla = c1 \\
		Avanzar(it)
	\end{WHILE}
	\comentario {recorro todos los destinos de c2 O(\#Casillas(t))}
\end{IF} \\
return \; res \\
\comentario{complejidad total: O(\#Casillas(t))}
\end{algorithm}

\begin{algorithm}{iSalidas}{\param{in}{t}{tablero(movimiento)}, \param{in}{c}{casilla}}{\res: conj(movimiento)}
\VAR datosCasilla : datosCasilla \\
datosCasilla \= Obtener(c, t.casillas) \comentario{O(1) por referencia} \\
\res \= datosCasilla.salidas \comentario{O(1) por referencia}  \\
return \, \res
\end{algorithm}

\begin{algorithm}{iOrigenesR}{\param{in}{t}{tablero($\mu$)}, \param{in}{c}{casilla}, \param{in}{m}{$\mu$}}{\res: conj(casilla)}
\res \= Vacio() \comentario{O(1)} \\
\VAR it : itConj(conj(casilla)) \\
it \= crearIt(iOrigenes(t, c, m)) \\
\VAR c' : casilla \\
\begin{WHILE}{HaySiguiente(it)}
	c' \= Siguiente(it) \\
	AgregarRapido(\res, c') \\
	Avanzar(it)\\
\end{WHILE}
\comentario{O(\#Origenes(t, c, m))} \\
\begin{IF}{EsVacio?(\res)}
	AgregarRapido(\res, c)
\end{IF} \\
return \; res \\
\end{algorithm}

\begin{algorithm}{iCasillasDe}{\param{in}{t}{tablero($\mu$)}, \param{in}{cc}{continente}}{\res: conj(casilla)}
\res \= Obtener(cc, t.continentes) \comentario{O(1) por referencia}
return \, \res
\end{algorithm}

\begin{algorithm}{iDestinos}{\param{in}{t}{tablero($\mu$)}, \param{in}{c}{casilla}}{\res: itConj(conj(tupla(casilla, movimiento)))}
\VAR datosCasilla : datosCasilla \\
datosCasilla \= Obtener(c, t.casillas) \\
\VAR destinos : conj(tupla(casilla, movimiento)) \\
destinos \= datosCasilla.destinos \\
\res \= crearIt(destinos) \comentario{O(1) por referencia al primer elemento} \\
return \, \res
\end{algorithm}

\begin{algorithm}{iProxCasilla}{\param{in}{t}{tablero(movimiento)}}{\res: casilla}
\res \= \#iCasillas(t) \comentario{O(1) tama�o del conj.}\\
return \, \res
\end{algorithm}

\begin{algorithm}{iProxContinente}{\param{in}{t}{tablero(movimiento)}}{\res: continente}
\res \= \#iContinentes(t) \comentario{O(1) tama�o del conj.}  \\
return \, \res
\end{algorithm}
