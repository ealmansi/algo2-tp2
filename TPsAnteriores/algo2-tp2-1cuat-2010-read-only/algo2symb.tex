% $Id: algo2symb.tex,v 1.6 2004/08/13 15:09:52 fpscha Exp $
\documentclass[a4paper]{article}
\usepackage[a4paper,margin=3.5cm,top=3.0cm,bottom=3.0cm]{geometry}
\usepackage[spanish,activeacute]{babel}
\usepackage{algo2symb}

\parskip=1.5ex
\pagestyle{empty}

\begin{document}
\RCS $Revision: 1.6 $
\RCSdate $Date: 2004/08/13 15:09:52 $

\begin{center}{\Large Package \texttt{algo2symb}} \par\medskip Versi'on \version -- \RCSDate\end{center}
\bigskip

Este paquete provee macros para diversos s'imbolos frecuentemente usados en documentos de la materia Algoritmos y Estructuras de Datos II (FCEyN, UBA). Los docentes de la c'atedra lo utilizamos para componer pr'acticas, apuntes y parciales; desde luego, los alumnos que decidan usar \LaTeX\ durante la cursada tambi'en est'an invitados a aprovecharlo.

Adicionalmente, este paquete carga \texttt{rcs}, \texttt{xspace}, \texttt{amssymb} y \texttt{amsmath}, y redefine el comando \verb|\implies| de estos 'ultimos para que produzca flecha corta $(\Rightarrow)$ en lugar de larga $(\Longrightarrow)$.

\bigskip\bigskip

\noindent \textbf{Listado de comandos provistos}
\medskip

Conectivos l'ogicos con \emph{luego}:

\begin{tabular}{l@{\hspace{36pt}}c}
\verb|\true| & \true \\
\verb|\false| & \false \\
\verb|\yluego| & \yluego \\
\verb|\oluego| & \oluego \\
\verb|\impluego| & \impluego
\end{tabular}
\medskip

S'imbolos relacionados con el TAD Secu:

\begin{tabular}{l@{\hspace{36pt}}c}
\verb|\secuvacia| & \secuvacia \\
\verb|\puntito| & \puntito \\
\verb|\circulito| & \circulito
\end{tabular}
\medskip

Notaci'on para complejidad (con par'entesis y \emph{math mode} impl'icitos):

\begin{tabular}{l@{\hspace{36pt}}l}
\verb|\Ode{n \cdot \log n}| & \Ode{n \cdot \log n} \\
\verb|\ode{n}| & \ode{n} \\
\verb|\Omegade{n^2}| & \Omegade{n^2} \\
\verb|\Thetade{n^3}| & \Thetade{n^3}
\end{tabular}
\medskip

Operadores con sub'indices en roman (no cursiva):

\begin{tabular}{l@{\hspace{36pt}}l}
\verb|\igobs| & \igobs\\
\verb|\igsub{abc}| & \igsub{abc} \\
\verb|\opsub{<}{xyz}| & \opsub{<}{xyz}
\end{tabular}
\medskip

Para indicar la presencia de un argumento en notaci'on infija o sufija ($\argumento \leq \argumento$):

\begin{tabular}{l@{\hspace{36pt}}c}
\verb|\argumento| & \argumento
\end{tabular}
\medskip

Para componer el nombre de un TAD:

\begin{tabular}{l@{\hspace{36pt}}l}
\verb|\nombretad{Nat}| & \nombretad{Nat} \\
\verb|\nombretad{Secu($\alpha$)}| & \nombretad{Secu($\alpha$)} \\
\verb|\nombretad{ColaB'ifida}| & \nombretad{ColaB'ifida}
\end{tabular}
\medskip

Otros s'imbolos:

\begin{tabular}{l@{\hspace{36pt}}c}
\verb|F: nat \en bool| & F: nat \en bool \\
\verb|\N| & \N \\
\verb|\cuidado| & \cuidado \\
\verb|\ax{3}| & \ax{3} \\
\verb|A \eqpor{\ax{3}} B| & A \eqpor{\ax{3}} B \\
\verb|$x \asig y+1$| & $x \asig y+1$ \\
\verb|\ssi| & \ssi \\
\verb|\sombrerito| & \sombrerito \\
\end{tabular}
\medskip

Macros para if\_then\_else\_fi (no es lo mejor, pero es lo que hay por ahora):

\verb|\lif guarda \lthen esto \lelse aquello \lfi| se ve as'i: \lif guarda \lthen esto \lelse aquello \lfi.

Tambi'en podemos recuadrar cosas con \verb|\recuadrar{Lo que sea.}|:

\recuadrar{Lo que sea.}



\bigskip

Nota: si bien por ahora \verb|\puntito| y \verb|\argumento| producen el mismo s'imbolo, la idea es usar distintos comandos para distinguirlos y que eventualmente pueda cambiarse alguno en el futuro (no es del todo feliz que sean el mismo).

Se intent'o buscar uno m'as peque'no para \verb|\puntito|, sin demasiado 'exito (salvo ``$\cdot$'', que parece \emph{demasiado} chiquitito). Adem'as, es agradable que \puntito\ y \circulito\ sean del mismo tama'no\ldots

\end{document}

