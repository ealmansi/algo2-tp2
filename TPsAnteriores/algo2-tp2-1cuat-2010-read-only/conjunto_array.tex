\newpage

\section{M'odulo Conjunto Acotado de Naturales}

\subsection{Aspectos de la interfaz}

\subsubsection{Servicios exportados}
Las operaciones exportadas tienen los siguientes 'ordenes de complejidad:
\newline

\orden{$\emptyset$(lower, upper)}{upper-lower}
\begin{itemize}
\item{Aliasing: }No se genera. 
\end{itemize}

\orden{Ag(a, ca)}{1}
\begin{itemize}
\item{Aliasing: }No se genera. 
\end{itemize}

\orden{$\emptyset?$}{1}
\begin{itemize}
\item{Aliasing: }No se genera. 
\end{itemize}

\orden{$\#$}{1}
\begin{itemize}
\item{Aliasing: }No se genera. 
\end{itemize}

\orden{\puntito -- \{ \puntito \}}{1}
\begin{itemize}
\item{Aliasing: }No se genera. 
\end{itemize}

%\orden{ca $\bigcup$ cb }{$\#$(ca) + $\#$(cb)}
%\begin{itemize}
%\item{Aliasing: }No se genera. 
%\end{itemize}

%\orden{ca $\bigcap$ cb}{min($\#$(ca), $\#$(cb))}
%\begin{itemize}
%\item{Aliasing: }No se genera. 
%\end{itemize}

\subsubsection{Interfaz}

\begin{interfaz}{ConjA}{\tadtitskip=4pt}

\usa{Bool, Nat, Conj($\alpha$), Arreglo\_dimensionable, Tupla}

\seexplicacon{ConjA}

\generos{conjA}

\exporta {generadores, observadores}

\vskip12pt

\operaciones
\newline
%$\emptyset$(lower, upper)
%Ag(a, ca)
%$\emptyset?$
%$\#$
%\puntito -- \{ \puntito \}

\textit{$\emptyset$} (in $lower$: nat, in $upper$: nat ) \en $res$: mapa
\\\tab \{$true$\}
\\\tab \{$res = Mapa(j)$\}
\newline

\textit{HabitacionFinal} (in $j$: juego) \en $res$: id
\\\tab \{$true$\}
\\\tab \{$res = HabitacionFinal(j)$\}
\newline

\end{interfaz}

\newpage

\subsection{Pautas de implementaci'on}

\subsubsection{Estructura de representaci'on}

\textbf{juego} se representa con \textbf{jugStruc}
\\\hspace*{1.1em} donde \textbf{jugStruc} es tupla
\\\tab $<$
\\\tab \textit{cardinal}: nat,
\\\tab \textit{estanBool}: ad(bool),
\\\tab \textit{conjunto}: conj(nat)
\\\tab $>$

\subsubsection{Invariante de representaci'on}
\begin{enumerate}
	\item{La segunda componente de cada elemento de infoJugador es una habitaci'on del tablero.}
\newline
\tab ($\forall$ c $\in$ [0 .. tam(e.infoJugador)) $\Pi_2$(e.infoJugador[c]) $\leq$ tam(e.quienEsta)
	\item{Cada habitaci�n distinta de la inicial tiene a lo sumo un jugador.}
\newline
\tab ($\forall$ c $\in$ [0 .. tam(e.quienEsta)) \#(e.quienEsta[c]) $\leq$ 1
	\item{Si un jugador j esta en la habitacion h, j pertenece al conjunto de jugadores correspondiente a la habitaci'on h en el arreglo quienEsta.}
\newline
\tab ($\forall$ j $\in$ [0 .. tam(e.infoJugador)) j == dameUno(e.quienEsta[$\Pi_2$(e.infoJugador[j]))
	\item{Toda llave que aparece en infoJugador, esta definida en llavesJugador.}
\newline
\tab	($\forall$ j $\in$ [0 .. tam(e.infoJugador)) ($\forall$ i $\in$ [0 .. \#($\Pi_1$(e.infoJugador[j]))) (*e.llavesJugador[k])[i] == 1
	\item{QuienEsta, Adyacentes, Llaves y Puertas tienen el mismo tama�o, pues estan indexados por habitaci'on.}
\newline
\tab (tam(e.quienEsta) == tam(e.adyacentes) == tam(e.llaves) == tam(e.puertas))
	\item{InfoJugador y LlavesJugador tienen el mismo tama�o, pues estan indexados por jugador.}
\newline
\tab (tam(e.infoJugador) = tam(e.llavesJugador))
	\item{Las llaves de un jugador en infoJugador son las mismas llaves de ese jugador en llavesJugador.}
\newline
\tab ($\forall$ j $\in$ [0 .. tam(e.infoJugador)))($\Pi_1$(e.infoJugador[j]) \igobs BoolToConj(e.llavesJugador[j]))
	\item{Cada uno de los arreglos apuntados por cada posici�n de e.adyacentes est� ordenado de manera ascendente.}
\newline
\tab ($\forall$ i $\in$ [0 .. tam(e.adyacentes))) ($\forall$ k $\in$ [0 .. tam(*e.adyacentes[i]))) ((*e.adyacentes[i])[k] $<$ (*e.adyacentes[i])[k+1])
	\item{La habitaci�n final es una habitaci�n del tablero.}
\newline
\tab e.habFinal $\in$ [0..tam(e.quienEsta))


\end{enumerate}

Rep : juego $\en$ bool

($\forall$ e : juego) Rep(e) $\equiv$
\newline

$ \hspace*{0.5em}
(1) \hspace*{1em}
\wedge \hspace*{1em}
(2) \hspace*{1em}
\wedge \hspace*{1em}
(3) \hspace*{1em}
\wedge \hspace*{1em}
(4) \hspace*{1em}
\wedge \hspace*{1em}
(5) \hspace*{1em}
\wedge \hspace*{1em}
(6) \hspace*{1em}
\wedge \hspace*{1em}
(7) \hspace*{1em}
\wedge \hspace*{1em}
(8) \hspace*{1em}
\wedge \hspace*{1em}
(9)
$
 \newline
 \newline
Aclaraci�n: Los elementos de e.puertas, e.llaves y e.adyacentes comparten las mismas caracter�sticas an�logamente que las puertas, llaves y habitaciones adyacentes de las habitaciones del tablero que usa el juego al momento de crearse, es decir, que las cosas que se piden en el invariante de representaci�n para tablero, se importan para los mismos elementos en la estructura de juego. Redondeando, como se cumple el rep de tablero cuando se pasa como par�metro y es �ste es convertido a un ''tablero est�tico inmodificable'', lo sigue cumpliendo.

\subsubsection{Funci'on de abstracci'on}

Abs : jugStruc e $\en$ tablero (Rep(e))

($\forall$ e : jugStruc) Abs(e) $=$ j : juego /
\newline

\tab \hspace*{0.7em} mapa(j) = mapa(e)

\tab $\wedge$ habitacionFinal(j) = longitud(e.quienEsta)

\tab $\wedge$ jugadores(j) = armarConjuntoAPartirDeArregloSegunIndices(e.infoJugadores)

\tab $\wedge$ llavesDe(j, jug) = $\Pi_1$(e.infojugador[jug]) $\vee$ BoolToConj(e.llavesDeJugador[jug])

\tab $\wedge$ posicionDe(j, jug) = $\Pi_2$(e.infojugador[jug])
\newline

\textbf{operaciones auxiliares}

\func{armarConjuntoAPartirDeArregloSegunIndices}{ad($\alpha$)}{conj(nat)}{}
\axioma{armarConjuntoAPartirDeArregloSegunIndices($a$)}{armoConjunto(longitud(a))}
\\
\func{armoConjunto}{nat}{conj(nat)}{}
\axioma{armoConjunto($0$)}{\{0\}}
\axioma{armoConjunto($n$)}{Ag(n, armoConjunto(n-1))}
\\
\func{BoolToConj}{ad[bool] a}{conj(nat)}{}
\axioma{BoolToConj(a[0])}{if a[0] == 1 then Ag(0, vacio) else vacio}
\axioma{BoolToConj(a)}{if a[tam(a)-1] == 1 then Ag((tam(a)-1), BoolToConj(subArreglo(a, 0,tam(a)-1))) else BoolToConj(subArreglo(a, 0, tam(a)-1))}

\newpage

\subsubsection{Algoritmos}

\begin{algorithm}{iMapa}{\param{in}{e}{jugStruc}}{res:tablero}
\VAR res : tablero \com{O(1)} \\
\VAR temp : ad(habitacion) \com{O(1)} \\
\VAR conjTemp : conj(habitacion) \com{O(1)} \\
res \= crear() \com{O(1)} \\
for(nat \; i = 0, i < MaxId, i++) \\
\tab	temp = *e.adyacentes[i] \\
\tab	for(nat \; k = 0, k < tam(*e.adyacentes[i]), k++) \\
\tab \tab		\begin{IF}{(k < i)}
							Ag(k, conjTemp)
						\end{IF} \\
\tab AgregarHabitacion(res, llaves[i], puertas[i], temp) \\
\tab conjTemp \= vacio
\end{algorithm}
\com{Complejidad Total: O(maxId * costo\_de\_insercion\_de\_habitacion). Ver m�dulo tablero.} \\

\begin{algorithm}{iHabitacionFinal}{\param{in}{e}{jugStruc}}{res:id}
res \= e.habFinal \com{O(1)}
\end{algorithm}
\com{Complejidad Total: O(1)} \\

\begin{algorithm}{iJugadores}{\param{in}{e}{jugStruc}}{res:conj(jugador) }
\VAR i : int  \com{O(1)} \\
\VAR cj: conj(jugador)  \com{O(1)} \\
for (i = 0, i < tam(e.infoJugador), i++)  \com{O(tam(e.infoJugador))} \\
\tab	Ag(i, cj) \\
res \= cj \com{O(tam(e.infoJugador))}
\end{algorithm}
\com{Complejidad Total: O(tam(e.infoJugador))} \\

\begin{algorithm}{iLlavesDe}{\param{in}{e}{jugStruc}, \param{in}{h}{jugador}}{res:conj(llave)}
res \= \Pi_1(e.infoJugador[h]) \com{O(1)}
\end{algorithm}
\com{Complejidad Total: O(1)} \\

\begin{algorithm}{iPosicionDe}{\param{in}{e}{jugStruc}, \param{in}{h}{jugador}}{res:id}
res \= \Pi_2(e.infoJugador[h]) \com{O(1)}
\end{algorithm}
\com{Complejidad Total: O(1)} \\

\begin{algorithm}{iFinalizo?}{\param{in}{e}{jugStruc}}{res:bool}
\VAR iter : itConj  \\
iter \= crearItConj(e.quienEsta[HabitacionFinal(e)]) \\
res \= haySiguiente(iter) \com{O(1)}
\end{algorithm}
\com{Complejidad Total: O(1)} \\

\begin{algorithm}{iOcupada?}{\param{in}{e}{jugStruc}, \param{in}{i}{id}}{res:bool}
\VAR iter : itConj  \\
iter \= crearItConj(e.quienEsta[h]) \\
res \= ((haySiguiente(iter) == true)) \com{O(1)}
\end{algorithm}
\com{Complejidad Total: O(1)} \\

\begin{algorithm}{iJugadorEn}{\param{in}{e}{jugStruc}, \param{in}{i}{id}}{res:jugador}
\VAR iter : itConj  \\
iter \= crearItConj(e.quienEsta[h]) \\
res \= actual(iter) \com{O(1)}
\end{algorithm}
\com{Complejidad Total: O(1)} \\

\begin{algorithm}{iPuedeMoverse}{\param{in}{e}{jugStruc}, \param{in}{j}{jugador}, \param{in}{i}{id}}{res:bool}
\VAR esta : id  \\
\VAR esAd : bool  \\
\VAR puedeAbrir : bool \\
\VAR tiene : bool  \\
\VAR iter : itConj  \\
\VAR buscar : int  \\
\VAR inf : int  \\
\VAR sup : int  \\
\VAR ady : ad[id] \\
esta \= PosicionDe(e,j) \com{O(1)} \\
ady \= *e.adyacentes[esta] \com{O(1)} \\
\com{Puertas, mira si puede abrir todas las puertas de la habitacion} \\
puedeAbrir \= false \com{O(1)} \\
tiene \= true \com{O(1)} \\
\com{la comparaci�n del pr�ximo if es O(1) porque el cardinal se obtiene en O(1), y el acceso a}\\
\com{tuplas es O(1) tambi�n. Luego, s�lo comparo por mayor o menor 2 naturales (O(1))}\\
\begin{IF}{(\#puertas[i] <= \#(\Pi_1(e.infoJugador[h])))}
	iter \= crearItConj(puertas(e.mapa,i)) \\
	\begin{WHILE}{(tieneProximo(iter) \wedge tiene)}
		tiene \= e.llavesJugador[actual(iter)] \com{O(1)}\\
		avanzarIt(iter)
	\end{WHILE} \\
	\com{O(\#(puertas))}
\ELSE{return \; false}
\end{IF} \\
\textbf{fi} \\
puedeAbrir \= tiene \com{O(1)} \\
\com{Busqueda Binaria} \\
esAd \= false \com{O(1)} \\
inf \= 0 \com{O(1)} \\
sup \= tam(e.adyacentes)-1 \com{O(1)} \\
\begin{WHILE}{(inf <= sup)\{}
	\com{O(log(x)) donde x es tam(e.adyacentes)} \\
	\com{Se va reduciendo siempre a la mitad el arreglo, o sea disminuye de a potencias de 2} \\
	buscar \= ((sup - inf) / 2) \com{Divisi'on entera} \\
	\begin{IF}{(ady[buscar] = h)}
		res \= true \com{O(1)}
		\ELSE
			\begin{IF}{(h < ady[buscar])}
				sup \= buscar - 1 \com{O(1)}
				\ELSE inf \= buscar + 1 \com{O(1)}
			\end{IF} \\
			\textbf{fi}
	\end{IF} \\
	\textbf{fi}
\end{WHILE} \\
\} \\
res \= puedeAbrir \wedge esAd \com{O(1)}
\end{algorithm} \\
\com{Complejidad Total: O(log(\#(e.adyacentes[i])) + \#(puertas)) siempre y cuando} \\
\com{haya a lo sumo misma cantidad de puertas que de llaves. Si hubiera mas puertas} \\
\com{que llaves, la comparaci'on es O(1) y ya no sigo con el algoritmo} \\

\begin{algorithm}{iMover}{\param{inout}{e}{jugStruc}, \param{in}{j}{jugador}, \param{in}{i}{id}}{}
\VAR iter : iterConj  \\
\VAR iterTemp : iterConj  \\
\VAR cj : conj(jugador)  \\
\com{Caso hay un jugador en donde se va a mover} \\
\begin{IF}{(i \neq 0 \yluego ocupada?(j,i))}  \com{no es la inicial y hay alguien, es un yluego} \\
	iterTemp \= crearIT(e.quienEsta[i]) \com{O(1)} \\
	\Pi_2(e.infoJugador[actual(iterTemp)]) \= 0 \com{O(1)} \\
	Ag(actual(iterTemp), e.quienEsta[0]) \com{O(1)} \\
	iter \= CrearItConj( \Pi_1(e.infoJugador[actual(iterTemp)]) ) \com{O(1)} \\
	\com{crea un iter de conjuntos sobre las llaves que tiene el jugador que esta en la habitacion.} \\
	\begin{WHILE}{(tieneProximoItConj(iter))}
	\com{ag llaves del jugador que estaba en la habitacion} \\
		Ag(actual(iter), \Pi_1(e.infoJugador[j])) \com{O(cantidad de llaves a copiar)} \\
		avanzarIt(iter) \com{O(1)}
	\end{WHILE} \\
	eliminarSiguiente(iterTemp) \com{SinUno(e.quienEsta[i])}
\end{IF} \\
\textbf{fi} \\
\com{ag llaves de la habitacion} \\
iter \= CrearItConj(llaves[i]) \com{O(1)} \\
\begin{WHILE}{(tieneProximoItConj(iter))}
	\com{O(cantidad de llaves a copiar de la habitaci'on)} \\
	Ag(actual(iter),\Pi_1(e.infoJugador[h])) \com{O(1)} \\
	avanzarItConj(iter) \com{O(1)}
\end{WHILE} \\
iter \= crearIt(\Pi_1(infoJugador[j])) \\
\begin{WHILE}{(tieneProximo(iter))}
	\com{O(cantidad de llaves a copiar de la habitaci'on)} \\
	llavesJugador[j][actual(iter)] \= true \com{O(1)} \\
	avanzarIt(iter) \com{O(1)} \\
\end{WHILE} \\
\com{saco al jugador de la casilla donde estaba.} \\
cj \= e.quienEsta[\Pi_2(e.infoJugador[h])] \com{O(1)} \\
iter \= CrearItConj(cj) \com{O(1)} \\
\begin{WHILE}{(tieneProximoItConj(iter))}
	\begin{IF}{(actual(iter) \neq h)}
		Ag(actual(iter),cj) \com{O(1)} \\
		avanzarItConj(iter) \com{O(1)} \\
	\end{IF} \\
	\textbf{fi}
\end{WHILE} \\
e.quienEsta[\Pi_2(e.infoJugador[h])] \= cj \\
\com{Muevo al jugador (cambio la casilla donde esta y lo posiciono en la casilla destino)} \\
Ag(h,e.quienEsta[i]) \\
\Pi_2(e.infoJugador[h]) \= i
\end{algorithm}
\com{Complejidad Total: O(\#(Llaves a copiar)) donde las llaves son las de la habitaci'on} \\
\com{de destino, y pueden o no sumarse las de otro jugador en el caso en en que hubiese otro} \\
\com{jugador en la habitaci'on de destino.} \\

\begin{algorithm}{iNuevo}{\param{in}{t}{tablero}, \param{in}{pos}{dicc(jugador, posicion)}, \param{in}{q}{id}}{res:jugStruc}
\VAR qE : ad[conj(jugador)]  \\
\VAR iJ : ad[tupla$<$conj(llaves),id$>$]  \\
\VAR aDy : ad[ad[id]]  \\
\VAR cJJ : conj(jugador)  \\
\VAR cJl : conj(llave)  \\
\VAR temp : ad[id]  \\
\VAR tupla : tupla$<$conj(llaves),id$>$  \\
\VAR lJ : ad[ad[bool]]  \\
\VAR keys : ad[bool]  \\
\VAR iter : iterConj  \\
\VAR suma : int  \\
\VAR temp2 : ad[int]  \\
\VAR contadores : ad[int] \\
\VAR lt : ad[conj(llave)] \\
\VAR cp : conj(puerta) \\
suma \= 0 \\
\com{Inicializar variables} \\
\com{inicializar iJ, lJ con tama�o \#claves(pos)} \\
iJ \= ad[\#claves(pos)] \com{O(\#(claves(pos)))} \\
lJ \= ad[\#claves(pos)] \com{O(\#(claves(pos)))} \\
\com{inicializar lt con tama�o MaxId} \\
lt \= ad[maxId] \com{O(maxId)}\\
for(nat \; k = 0, k < maxId, k++) \\
\tab iter \= crearIt(llaves(k, t) \com{k es la habitacion actual, t es el tablero que nos pasan como parametro} \\
\tab \begin{WHILE}{(tieneProximo(iter))}
\tab Ag(actual(iter),cj) \\
\tab proximo(iter) \\
\tab \com{O($\sum_{k=0}^{k=maxId}{llaves(k, t)}$)}
\end{WHILE} \\
\tab lt[k] \= cj \\
cj \= vacio \\
\com{inicializar pt con tama�o MaxId} \\
pt \= ad[maxId] \com{O(maxId)}\\
for(nat \; k = 0, k < maxId, k++) \\
\tab iter \= crearIt(puertas(k, t) \com{k es la habitacion actual, t es el tablero que nos pasan como parametro} \\
\tab \begin{WHILE}{(tieneProximo(iter))}
\tab Ag(actual(iter),cj) \\
\tab proximo(iter) \\
\tab \com{O($\sum_{k=0}^{k=maxId}{puertas(k, t)}$)}
\end{WHILE} \\
\tab pt[k] \= cj \\
\com{inicializar qE con tama�o i+1, donde i es la ultima id de habitacion,} \\
\com{Tiene que ser uno mas grande el tama�o} \\
qE \= ad[i+1] \com{O(i)}\\
\com{inicializar adY con tama�o i+1 donde i es la ultima id de habitacion,} \\
\com{Tiene que ser uno mas grande el tama�o} \\
aDy \= ad[i+1] \com{O(i)}\\
\com{ya inicializamos cJJ, cJl vacios} \\
\com{cargar posiciones de los jugadores en iJ y en habitaciones (qE)} \\
for(j = 0, j < tam(iJ)-1, j++) \\
\tab	\Pi_2(iJ[j]) \= obtener(j,pos) \com{supongo es por copia, obtener es O(1)} \\
\tab	qE[obtener(j,pos)] \= j \com{O(cantidad de jugadores)} \\
\com{cargar llaves de jugadores en iJ y en lJ} \\
\com{inicializar en todas las posiciones de lJ un array de tama�o maxid llaves,} \\
\com{en false todos los valores de los arrays de adentro} \\
\com{todo eso es O(max(\#(Llaves))*cantidad de jugadores)} \\
for(j = 0, j < tam(iJ) - 1, j++ ) \com{lleno los conjuntos} \\
\tab	\Pi_1(iJ[j]) \= llaves(t,j) \com{por copia} \\
for(j = 0, j < tam(lJ) - 1, j++ ) \com{lleno el array} \\
\tab	iter \= crearIt(\Pi_1(iJ[j])) \\
\tab	\begin{WHILE}{(tieneProximoIt(iter))}
					lJ[j][actual(iter)] \= true \\
					avanzarIt(iter)
			\end{WHILE} \\
\com{llenar adyacentes} \\
\com{inicializar arrays, cada uno de tama�o cantidad de adyacentes propias)} \\
for(j = 0, j < i, j++) \\
\tab \com{O($\sum_{h=0}^{maxIdHab}{\#(adyacentes(h))}$)} \\
\tab 	suma \= \#sucesoras(t,j) \com{O(1)} \\
\tab 	\begin{IF}{(j \neq 0)}
				suma \= suma + \#predecesoras(t,j) \com{O(1)} \\
			\end{IF} \\
\com{inicializar temp con tama�o = suma, y poner el array en la posicion aDy[j]} \\
	\tab temp \= *ad[suma] \com{O(i)}\\
	\tab aDy[j] \= temp \\
\com{llenar los arrays} \\
contadores \= ad[i] \\
for(a=0, a<i, a++) \\
\tab contadores[a] \=0 \com{O(1)} \\
for( j = 0, j < i, j++) \\
\tab	\begin{IF}{(j \neq 0)} 
				iter \= crearItConj(predecesoras(t,j)) \com{O(1)} \\
				\begin{WHILE}{(tieneProximo(iter))}		
					\com{O(\#predecesoras(habitaci'on))} \\
					*aDy[actual(iter)][contadores[actual(iter)]] \= j \com{VER ACLARACION} \\
					contadores[actual(iter)] \= contadores[actual(iter)] + 1 \com{O(1)} \\
					avanzar(iter) \com{O(1)}
				\end{WHILE}
			\end{IF} \\
iter \= crearItConj(sucesoras(t,j)) \com{O(1)} \\
\begin{WHILE}{(tieneProximo(iter))}
	\com{O(\#sucesoras(habitaci'on))} \\
	*aDy[actual(iter)][contadores[actual(iter)]] \= j \com{VER ACLARACION} \\
	contadores[actual(iter)] \= contadores[actual(iter)] + 1 \com{O(1)} \\
	avanzar(iter) \com{O(1)}
\end{WHILE} \\ \\
res.quienEsta \= qE \com{O(1) por referencia} \\
res.infoJugador \= iJ \com{O(1) por referencia} \\
res.adyacentes \= aDy \com{O(1) por referencia} \\
res.llavesJugador \= lJ \com{O(1) por referencia} \\
res.llaves \= lt \com{O(1) por referencia} \\
res.puertas \= lp \com{O(1) por referencia} \\
res.habFinal \= q \com{O(1) por copia por ser tipo b�sico} \\
\com{*****ACLARACION, hay que agregarlo en la primera posicion disponible del array,} \\
\com{en O(1), para eso tengo q guardar un entero con el array que me diga la cantidad de} \\
\com{elementos q tiene ocupados ya, hay q hacer q el array tenga eso. Otra opcion seria} \\
\com{crear otro array temporal, que sea ad[ad[int]] que lleve la cuenta de cuantos} \\
\com{van agreagados ya.} \\
\end{algorithm}
\com{Complejidad Total: O(MaxIdHabitaciones(t) + m + max(Llaves(t))*max(claves(pos)))} \\
\com{con m = $\displaystyle\sum_{h \in H}{(\#(adyacentes(tab, pos)) + \#(LlavesDe(tab, jug)) + \#(Puertas(tab, jug)))}$}
\section*{Fin M'odulo Juego}