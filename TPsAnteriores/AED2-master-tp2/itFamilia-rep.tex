\begin{center}
\begin{tabular}{|l|} 
\hline
\\
itFamilia se representa con puntero(DatosCat)\\\\

\hspace*{6em}datosCat es \tupla{\\
\hspace*{6em}\param{}{categoria}{Categoria},\\ 
\hspace*{6em}\param{}{id}{nat},\\ 
\hspace*{6em}\param{}{altura}{nat},\\ 
\hspace*{6em}\param{}{hijos}{Conj(puntero(datosCat))},\\
\hspace*{6em}\param{}{padre}{puntero(datosCat)}}\\ \\ 


\hline
\end{tabular}
\end{center}

\par itFamilia es un iterador de puntero a datoscat que al hacer siguiente va al puntero datoscat padre. Al manejarse con punteros sus complejidades son O(1).

\subsubsection{Invariante de Representaci\'on}
%\paragraph{El Invariante Informalmente}
%\begin{enumerate}
%\item Lo iterado por el iterador de mensajes recientes es válido.
%\end{enumerate}

\paragraph{El Invariante Formalmente}
{\begin{changemargin}{0em}{4em}Rep : estrITF $\rightarrow$ boolean\par \end{changemargin}(\paratodo\param{}{it}{estrITF})} Rep(it) $\equiv$ true

\subsubsection{Funci\'on de Abstracci\'on}
\begin{ABS}{e}{estrITF}{it}{itUni(Categoria)}
\absfunc{siguientePadres(e) \igobs siguientes(it)}{}
\end{ABS}

\paragraph{Funciones auxiliares}
\paragraph*{}

\hspace*{1em}
\tadOperacion{siguientePadres}{estrITF e}{secu(tupla<Categoria, nat>)}{}
\tadAxioma{siguientePadres(e)}{\IF (*e).padre == NULL THEN <(*e).padre.categoria, (*e).padre.id> $\bullet$ $\emptyset$ ELSE <(*e).padre.categoria, (e*).padre.id> $\bullet$ siguientePadres((*e).padre) FI}
