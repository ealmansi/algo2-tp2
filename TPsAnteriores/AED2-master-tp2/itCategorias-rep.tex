\begin{center}
\begin{tabular}{|l|} 
\hline
\\
itCategorias se representa con itLista(datosCat)\\\\

\hspace*{6em}datosCat es \tupla{\\
\hspace*{6em}\param{}{categoria}{Categoria},\\ 
\hspace*{6em}\param{}{id}{nat},\\ 
\hspace*{6em}\param{}{altura}{nat},\\ 
\hspace*{6em}\param{}{hijos}{Conj(puntero(datosCat))},\\
\hspace*{6em}\param{}{padre}{puntero(datosCat)}}\\ \\ 

\hline
\end{tabular}
\end{center}

\par itCategorias es un iterador de lista común. Sus complejidades nos alcanzan para iterar una Lista(datosCat).

\subsubsection{Invariante de Representaci\'on}
%\paragraph{El Invariante Informalmente}
%\begin{enumerate}
%\item Lo iterado por el iterador de mensajes recientes es válido.
%\end{enumerate}

\paragraph{El Invariante Formalmente}
{\begin{changemargin}{0em}{4em}Rep : estrITC $\rightarrow$ boolean\par \end{changemargin}(\paratodo\param{}{it}{estrITC}) Rep(it) $\equiv$ true

\subsubsection{Funci\'on de Abstracci\'on}
\begin{ABS}{e}{estrITC}{it}{itUni(Categoria)}
\absfunc{siguientes(e) \igobs siguientes(it)}{}
\end{ABS}