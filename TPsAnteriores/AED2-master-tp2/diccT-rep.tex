\begin{center}
\begin{tabular}{|l|} 
\hline
\\
DiccTrie($\alpha$) \textbf{se representa con} estrDT, donde estrDT es Puntero(Nodo)\\
\\
\hspace*{6em}Nodo es \tupla{\param{}{arreglo}{arreglo(Puntero(Nodo))[256]}, \param{}{significado}{Puntero($\alpha$)}}\\
\\
\hline
\end{tabular}
\end{center}

\par La estructura es un puntero a Nodo en la cual cada nodo es una tupla entre un arreglo y un significado para el dicc. Cada posición del arreglo representa una letra y su contenido es un puntero al nodo de la letra siguiente o a Null.

\subsubsection{Invariante de Representaci\'on}
\paragraph{El Invariante Informalmente}
\begin{enumerate}
\item No hay repetidos en arreglo de Nodo salvo por Null. Todas las posiciones del arreglo están definidas.
\item No se puede volver al Nodo actual siguiendo alguno de los punteros hijo del actual o de alguno de los hijos de estos.
\item O bien el Nodo es una hoja, o todos sus punteros hijo no-nulos llevan a hojas siguiendo su recorrido.
\end{enumerate}

\paragraph{El Invariante Formalmente}
\paragraph*{}
\begin{Rep}{estrDT}{e}
\repfunc{(\forall\param{}{i,j}{nat}) 0 \leq i \leq 255  \land 0 \leq j \leq 255 \Rightarrow \\
Definido?((*e).Arreglo,i) \land Definido?((*e).Arreglo,j) \land \\
(i=j) \lor \\
(i \neq j \land ((*e).Arreglo[i] = null \land (*e).Arreglo[j] = null \lor \\
(*e).Arreglo[i] \neq (*e).Arreglo[j])}{\yluego}
\repfunc{(\neg\exists\param{}{n}{nat}) EncAEstrDTEnNMov(e,e,n)}{\yluego}
\repfunc{SonTodosNullOLosHijosLoSon(e)}{}
\end{Rep}

\subsubsection{Funci\'on de Abstracci\'on}
\begin{ABS}{e}{estrDT}{d}{diccT(c,$\alpha$)}
\absfunc{(\forall\param{}{c}{clave})def?(c,d) \igobs estaDefinido?(c,e)}{\yluego}
\absfunc{(\forall\param{}{h}{clave})def?(h,d) \Rightarrow obtener(h,d) \igobs ObtenerS(h,*(e))}{}
\end{ABS}

\paragraph{Funciones auxiliares}
\paragraph*{}

\hspace*{1em}\tadOperacion{EncAEstrDTEnNMov}{estrDT, estrDT, Nat}{Bool}{}
\tadAxioma{EncAEstrDTEnNMov(buscado,actual,n)}{\IF (n = 0) THEN EstaEnElArregloActual?(buscado,actual,255) ELSE RecurrenciaConLosHijos(buscado,actual, n-1,255) FI}
\vspace*{1em}

\hspace*{1em}\tadOperacion{EstaEnElArregloActual?}{estrDT, estrDT, nat}{Bool}{}
\tadAxioma{EstaEnElArregloActual?(buscado,actual,n)}{\IF (n=0) THEN ((*actual).Arreglo[0] = buscado) ELSE ((*actual).Arreglo[n] = buscado) $\lor$ (EstaEnElArregloActual? (buscado,actual,n-1)) FI}
\vspace*{1em}

\hspace*{1em}\tadOperacion{RecurrenciaConLosHijos}{estrDT, estrDT, nat, nat}{Bool}{}
\tadAxioma{RecurrenciaConLosHijos(buscado,actual,n,i)}{\IF (i = 0) THEN EncAEstrDTEnNMov(buscado,(*actual).Arreglo[0],n) ELSE EncAEstrDTEnNMov(buscado, (*actual).Arreglo[i],n) $\lor$ (RecurrenciaConLosHijos(buscado,actual,n,i-1) FI}
\vspace*{1em}

\hspace*{1em}\tadOperacion{SonTodosNullOLosHijosLoSon}{estrDT}{Bool}{}
\tadAxioma{SonTodosNullOLosHijosLoSon(e)}{Los256SonNull(e,255) $\lor$ BuscarHijosNull (e, 255)}

\hspace*{1em}\tadOperacion{Los256SonNull}{estrDT, nat}{Bool}{}
\tadAxioma{Los256SonNull(e,i)}{\IF (i = 0) THEN ((*e).Arreglo[0] = null) ELSE ((*e).Arreglo[i] = null) $\land$ Los256SonNull(e, i-1) FI}
\vspace*{1em}

\hspace*{1em}\tadOperacion{BuscarHijosNull}{estrDT, nat}{Bool}{}
\tadAxioma{BuscarHijosNull(e,i)}{\IF (i = 0) THEN ((*e).Arreglo[0] = null) $\lor{_L}$ SonTodosNullOLosHijosLoSon((*e).Arreglo[0]) ELSE (((*e).Arreglo[i] = null) $\lor{_L}$ SonTodosNullOLosHijosLoSon((*e).Arreglo[i])) $\land$ BuscarHijosNull(e,i-1) FI}

\hspace*{1em}\tadOperacion{estaDefinido?}{string, estrDT}{bool}{}
\tadAxioma{estaDefinido?(c,e)}{\IF (e==Null) THEN false ELSE NodoDef?(c,*(e)) FI}

\hspace*{1em}\tadOperacion{NodoDef?}{string, Nodo}{bool}{}
\tadAxioma{NodoDef?(c,n)}{\IF (vacia?(c)) THEN true ELSE {\IF (n.arreglo[ord(prim(c))] $\neq$ Null) THEN NodoDef?(fin(c),*(n.arreglo[ord(prim(c))]))ELSE false FI} FI}

\hspace*{1em}\tadOperacion{ObtenerS}{string, Nodo}{$\alpha$}{}
\tadAxioma{ObtenerS(c,n)}{\IF (vacia?(c)) THEN *(n.significado) ELSE ObtenerS(fin(c),*(n.arreglo[ord(prim(c))]))FI}


